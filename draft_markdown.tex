% Options for packages loaded elsewhere
\PassOptionsToPackage{unicode}{hyperref}
\PassOptionsToPackage{hyphens}{url}
%
\documentclass[
]{article}
\usepackage{amsmath,amssymb}
\usepackage{iftex}
\ifPDFTeX
  \usepackage[T1]{fontenc}
  \usepackage[utf8]{inputenc}
  \usepackage{textcomp} % provide euro and other symbols
\else % if luatex or xetex
  \usepackage{unicode-math} % this also loads fontspec
  \defaultfontfeatures{Scale=MatchLowercase}
  \defaultfontfeatures[\rmfamily]{Ligatures=TeX,Scale=1}
\fi
\usepackage{lmodern}
\ifPDFTeX\else
  % xetex/luatex font selection
\fi
% Use upquote if available, for straight quotes in verbatim environments
\IfFileExists{upquote.sty}{\usepackage{upquote}}{}
\IfFileExists{microtype.sty}{% use microtype if available
  \usepackage[]{microtype}
  \UseMicrotypeSet[protrusion]{basicmath} % disable protrusion for tt fonts
}{}
\makeatletter
\@ifundefined{KOMAClassName}{% if non-KOMA class
  \IfFileExists{parskip.sty}{%
    \usepackage{parskip}
  }{% else
    \setlength{\parindent}{0pt}
    \setlength{\parskip}{6pt plus 2pt minus 1pt}}
}{% if KOMA class
  \KOMAoptions{parskip=half}}
\makeatother
\usepackage{xcolor}
\usepackage[margin=1in]{geometry}
\usepackage{color}
\usepackage{fancyvrb}
\newcommand{\VerbBar}{|}
\newcommand{\VERB}{\Verb[commandchars=\\\{\}]}
\DefineVerbatimEnvironment{Highlighting}{Verbatim}{commandchars=\\\{\}}
% Add ',fontsize=\small' for more characters per line
\usepackage{framed}
\definecolor{shadecolor}{RGB}{248,248,248}
\newenvironment{Shaded}{\begin{snugshade}}{\end{snugshade}}
\newcommand{\AlertTok}[1]{\textcolor[rgb]{0.94,0.16,0.16}{#1}}
\newcommand{\AnnotationTok}[1]{\textcolor[rgb]{0.56,0.35,0.01}{\textbf{\textit{#1}}}}
\newcommand{\AttributeTok}[1]{\textcolor[rgb]{0.13,0.29,0.53}{#1}}
\newcommand{\BaseNTok}[1]{\textcolor[rgb]{0.00,0.00,0.81}{#1}}
\newcommand{\BuiltInTok}[1]{#1}
\newcommand{\CharTok}[1]{\textcolor[rgb]{0.31,0.60,0.02}{#1}}
\newcommand{\CommentTok}[1]{\textcolor[rgb]{0.56,0.35,0.01}{\textit{#1}}}
\newcommand{\CommentVarTok}[1]{\textcolor[rgb]{0.56,0.35,0.01}{\textbf{\textit{#1}}}}
\newcommand{\ConstantTok}[1]{\textcolor[rgb]{0.56,0.35,0.01}{#1}}
\newcommand{\ControlFlowTok}[1]{\textcolor[rgb]{0.13,0.29,0.53}{\textbf{#1}}}
\newcommand{\DataTypeTok}[1]{\textcolor[rgb]{0.13,0.29,0.53}{#1}}
\newcommand{\DecValTok}[1]{\textcolor[rgb]{0.00,0.00,0.81}{#1}}
\newcommand{\DocumentationTok}[1]{\textcolor[rgb]{0.56,0.35,0.01}{\textbf{\textit{#1}}}}
\newcommand{\ErrorTok}[1]{\textcolor[rgb]{0.64,0.00,0.00}{\textbf{#1}}}
\newcommand{\ExtensionTok}[1]{#1}
\newcommand{\FloatTok}[1]{\textcolor[rgb]{0.00,0.00,0.81}{#1}}
\newcommand{\FunctionTok}[1]{\textcolor[rgb]{0.13,0.29,0.53}{\textbf{#1}}}
\newcommand{\ImportTok}[1]{#1}
\newcommand{\InformationTok}[1]{\textcolor[rgb]{0.56,0.35,0.01}{\textbf{\textit{#1}}}}
\newcommand{\KeywordTok}[1]{\textcolor[rgb]{0.13,0.29,0.53}{\textbf{#1}}}
\newcommand{\NormalTok}[1]{#1}
\newcommand{\OperatorTok}[1]{\textcolor[rgb]{0.81,0.36,0.00}{\textbf{#1}}}
\newcommand{\OtherTok}[1]{\textcolor[rgb]{0.56,0.35,0.01}{#1}}
\newcommand{\PreprocessorTok}[1]{\textcolor[rgb]{0.56,0.35,0.01}{\textit{#1}}}
\newcommand{\RegionMarkerTok}[1]{#1}
\newcommand{\SpecialCharTok}[1]{\textcolor[rgb]{0.81,0.36,0.00}{\textbf{#1}}}
\newcommand{\SpecialStringTok}[1]{\textcolor[rgb]{0.31,0.60,0.02}{#1}}
\newcommand{\StringTok}[1]{\textcolor[rgb]{0.31,0.60,0.02}{#1}}
\newcommand{\VariableTok}[1]{\textcolor[rgb]{0.00,0.00,0.00}{#1}}
\newcommand{\VerbatimStringTok}[1]{\textcolor[rgb]{0.31,0.60,0.02}{#1}}
\newcommand{\WarningTok}[1]{\textcolor[rgb]{0.56,0.35,0.01}{\textbf{\textit{#1}}}}
\usepackage{graphicx}
\makeatletter
\def\maxwidth{\ifdim\Gin@nat@width>\linewidth\linewidth\else\Gin@nat@width\fi}
\def\maxheight{\ifdim\Gin@nat@height>\textheight\textheight\else\Gin@nat@height\fi}
\makeatother
% Scale images if necessary, so that they will not overflow the page
% margins by default, and it is still possible to overwrite the defaults
% using explicit options in \includegraphics[width, height, ...]{}
\setkeys{Gin}{width=\maxwidth,height=\maxheight,keepaspectratio}
% Set default figure placement to htbp
\makeatletter
\def\fps@figure{htbp}
\makeatother
\setlength{\emergencystretch}{3em} % prevent overfull lines
\providecommand{\tightlist}{%
  \setlength{\itemsep}{0pt}\setlength{\parskip}{0pt}}
\setcounter{secnumdepth}{-\maxdimen} % remove section numbering
\usepackage{multirow}
\usepackage{multicol}
\usepackage{colortbl}
\usepackage{hhline}
\newlength\Oldarrayrulewidth
\newlength\Oldtabcolsep
\usepackage{longtable}
\usepackage{array}
\usepackage{hyperref}
\usepackage{float}
\usepackage{wrapfig}
\ifLuaTeX
  \usepackage{selnolig}  % disable illegal ligatures
\fi
\IfFileExists{bookmark.sty}{\usepackage{bookmark}}{\usepackage{hyperref}}
\IfFileExists{xurl.sty}{\usepackage{xurl}}{} % add URL line breaks if available
\urlstyle{same}
\hypersetup{
  pdftitle={The Code for `Fruit flies as bioindicators: Drosophila species distribution relative to environmental disturbance'},
  pdfauthor={700041182},
  hidelinks,
  pdfcreator={LaTeX via pandoc}}

\title{The Code for `Fruit flies as bioindicators: Drosophila species
distribution relative to environmental disturbance'}
\author{700041182}
\date{2024-03-19}

\begin{document}
\maketitle

\hypertarget{organising-the-data}{%
\section{Organising the Data}\label{organising-the-data}}

First, we ensure the necessary packages are installed and loaded.

\begin{Shaded}
\begin{Highlighting}[]
\FunctionTok{library}\NormalTok{(pacman)}
\FunctionTok{p\_load}\NormalTok{(dplyr, tidyr, lubridate, tibble, vegan, tabula, lme4, flextable, readr, ggplot2, patchwork)}
\end{Highlighting}
\end{Shaded}

Next, we load in the raw \emph{Drosophila} data by setting the pathway
for the file and reading it in.

\begin{Shaded}
\begin{Highlighting}[]
\NormalTok{dros\_path }\OtherTok{\textless{}{-}} \StringTok{"./data/raw\_data\_drosophila.csv"}
\NormalTok{dros\_data }\OtherTok{\textless{}{-}} \FunctionTok{read\_csv}\NormalTok{(dros\_path, }\AttributeTok{col\_select =} \SpecialCharTok{{-}}\FunctionTok{c}\NormalTok{(}\DecValTok{7}\NormalTok{, }\DecValTok{8}\NormalTok{))}
\end{Highlighting}
\end{Shaded}

\begin{verbatim}
## New names:
## Rows: 2021 Columns: 6
## -- Column specification
## -------------------------------------------------------- Delimiter: "," chr
## (4): area_type, trap_no, species, notes date (2): refill_date, collect_date
## i Use `spec()` to retrieve the full column specification for this data. i
## Specify the column types or set `show_col_types = FALSE` to quiet this message.
## * `` -> `...7`
## * `` -> `...8`
\end{verbatim}

By listing the unique species, we can see that there were some spelling
mistakes.

\begin{Shaded}
\begin{Highlighting}[]
\FunctionTok{list}\NormalTok{(}\FunctionTok{unique}\NormalTok{(dros\_data}\SpecialCharTok{$}\NormalTok{species))}
\end{Highlighting}
\end{Shaded}

\begin{verbatim}
## [[1]]
##  [1] "immi" "mel"  "sub"  "suz"  "obs"  "tris" "hyd"  "fun"  NA     "subs"
## [11] "bus"
\end{verbatim}

Here, we correct these.

\begin{Shaded}
\begin{Highlighting}[]
\NormalTok{dros\_data}\SpecialCharTok{$}\NormalTok{species[dros\_data}\SpecialCharTok{$}\NormalTok{species }\SpecialCharTok{==} \StringTok{"subs"}\NormalTok{] }\OtherTok{\textless{}{-}} \StringTok{"sub"}
\end{Highlighting}
\end{Shaded}

Trap 1 was not included in the final analyses, so we remove this from
the data frame.

\begin{Shaded}
\begin{Highlighting}[]
\NormalTok{dros\_data }\OtherTok{\textless{}{-}}\NormalTok{ dros\_data[dros\_data}\SpecialCharTok{$}\NormalTok{trap\_no }\SpecialCharTok{!=} \StringTok{"D1"}\NormalTok{, ]}
\end{Highlighting}
\end{Shaded}

Now, we load the GPS data. We set the pathway for the GPS data and
reading it in.

\begin{Shaded}
\begin{Highlighting}[]
\NormalTok{coord\_path }\OtherTok{\textless{}{-}} \StringTok{"./data/site\_coords.csv"}
\NormalTok{coord\_data }\OtherTok{\textless{}{-}} \FunctionTok{read\_csv}\NormalTok{(coord\_path, }\AttributeTok{col\_select =} \SpecialCharTok{{-}}\FunctionTok{c}\NormalTok{(}\DecValTok{4}\NormalTok{))}
\end{Highlighting}
\end{Shaded}

\begin{verbatim}
## New names:
## Rows: 11 Columns: 3
## -- Column specification
## -------------------------------------------------------- Delimiter: "," chr
## (1): trap_no dbl (2): lat, long
## i Use `spec()` to retrieve the full column specification for this data. i
## Specify the column types or set `show_col_types = FALSE` to quiet this message.
## * `` -> `...4`
\end{verbatim}

Next, we merge the GPS coordinates to the existing dataframe, adding
latitude and longitude (`lat' and `long') columns.

\begin{Shaded}
\begin{Highlighting}[]
\NormalTok{codros\_data }\OtherTok{\textless{}{-}}\NormalTok{ dros\_data }\SpecialCharTok{\%\textgreater{}\%} \FunctionTok{left\_join}\NormalTok{(coord\_data, }\AttributeTok{by =} \StringTok{"trap\_no"}\NormalTok{)}
\end{Highlighting}
\end{Shaded}

For the PERMANOVA analysis, we require the data frame to be in a widened
format. Here, we pivot the data so that we have columns for each
detected species, quantifying the total abundance of each species across
the traps.

\begin{Shaded}
\begin{Highlighting}[]
\NormalTok{trap\_no\_abundance }\OtherTok{\textless{}{-}}\NormalTok{ total\_site\_abundance\_df\_wide }\OtherTok{\textless{}{-}}\NormalTok{ dros\_data }\SpecialCharTok{\%\textgreater{}\%}
  \FunctionTok{group\_by}\NormalTok{(species, trap\_no, area\_type) }\SpecialCharTok{\%\textgreater{}\%}
  \FunctionTok{summarise}\NormalTok{(}\AttributeTok{count =} \FunctionTok{n}\NormalTok{()) }\SpecialCharTok{\%\textgreater{}\%}
  \FunctionTok{pivot\_wider}\NormalTok{(}\AttributeTok{names\_from =}\NormalTok{ species, }\AttributeTok{values\_from =}\NormalTok{ count, }\AttributeTok{values\_fill =} \DecValTok{0}\NormalTok{)}
\end{Highlighting}
\end{Shaded}

\begin{verbatim}
## `summarise()` has grouped output by 'species', 'trap_no'. You can override
## using the `.groups` argument.
\end{verbatim}

For the sake of the PERMANOVA, we do not want `NA' to be counted as a
species column, so we remove the `NA' column.

\begin{Shaded}
\begin{Highlighting}[]
\NormalTok{trap\_no\_abundance\_noNA }\OtherTok{\textless{}{-}}\NormalTok{ trap\_no\_abundance[, }\FunctionTok{colnames}\NormalTok{(trap\_no\_abundance) }\SpecialCharTok{!=} \StringTok{"NA"}\NormalTok{]}
\end{Highlighting}
\end{Shaded}

\hypertarget{analysis}{%
\section{Analysis}\label{analysis}}

\hypertarget{permanova}{%
\subsubsection{PERMANOVA}\label{permanova}}

Before the PERMANOVA is ran, we fourth-root transform the data.

\begin{Shaded}
\begin{Highlighting}[]
\NormalTok{trans\_trap\_no\_abundance\_noNA }\OtherTok{\textless{}{-}}\NormalTok{ trap\_no\_abundance\_noNA }\SpecialCharTok{\%\textgreater{}\%}
  \FunctionTok{mutate\_if}\NormalTok{(is.numeric, }\ControlFlowTok{function}\NormalTok{(x) x}\SpecialCharTok{\^{}}\NormalTok{(}\DecValTok{1}\SpecialCharTok{/}\DecValTok{4}\NormalTok{))}
\end{Highlighting}
\end{Shaded}

\begin{verbatim}
## `mutate_if()` ignored the following grouping variables:
## * Column `trap_no`
\end{verbatim}

To create the Bray-Curtis dissimilarity matrix, we must remove the
non-numeric columns.

\begin{Shaded}
\begin{Highlighting}[]
\NormalTok{trans\_trap\_numeric }\OtherTok{\textless{}{-}}\NormalTok{ trans\_trap\_no\_abundance\_noNA[, }\FunctionTok{sapply}\NormalTok{(trans\_trap\_no\_abundance\_noNA, is.numeric)]}
\end{Highlighting}
\end{Shaded}

Then, we can create the Bray-Curtis dissimilarity matrix and ensure that
it is stored as a matrix.

\begin{Shaded}
\begin{Highlighting}[]
\NormalTok{trans\_trap\_dist\_matrix }\OtherTok{\textless{}{-}} \FunctionTok{vegdist}\NormalTok{(trans\_trap\_numeric, }\AttributeTok{method =} \StringTok{"bray"}\NormalTok{)}
\NormalTok{trans\_trap\_dist\_matrix }\OtherTok{\textless{}{-}} \FunctionTok{as.matrix}\NormalTok{(trans\_trap\_numeric)}
\end{Highlighting}
\end{Shaded}

Finally, we can run the PERMANOVA.

\begin{Shaded}
\begin{Highlighting}[]
\NormalTok{trans\_trap\_perm\_result }\OtherTok{\textless{}{-}} \FunctionTok{adonis2}\NormalTok{(trans\_trap\_dist\_matrix }\SpecialCharTok{\textasciitilde{}}\NormalTok{ area\_type, }\AttributeTok{data =}\NormalTok{ trans\_trap\_no\_abundance\_noNA)}

\NormalTok{trans\_trap\_perm\_result}
\end{Highlighting}
\end{Shaded}

\begin{verbatim}
## Permutation test for adonis under reduced model
## Terms added sequentially (first to last)
## Permutation: free
## Number of permutations: 999
## 
## adonis2(formula = trans_trap_dist_matrix ~ area_type, data = trans_trap_no_abundance_noNA)
##           Df SumOfSqs      R2      F Pr(>F)
## area_type  1 0.032674 0.17998 1.7559  0.184
## Residual   8 0.148866 0.82002              
## Total      9 0.181540 1.00000
\end{verbatim}

\hypertarget{species-abundance-t-tests}{%
\subsubsection{Species Abundance
T-Tests}\label{species-abundance-t-tests}}

First we must run Shapiro-Wilks tests for normality of the data.

\begin{Shaded}
\begin{Highlighting}[]
\FunctionTok{shapiro.test}\NormalTok{(trap\_no\_abundance\_noNA}\SpecialCharTok{$}\NormalTok{hyd[trans\_trap\_no\_abundance\_noNA}\SpecialCharTok{$}\NormalTok{area\_type}\SpecialCharTok{==}\StringTok{"N"}\NormalTok{])}
\end{Highlighting}
\end{Shaded}

\begin{verbatim}
## 
##  Shapiro-Wilk normality test
## 
## data:  trap_no_abundance_noNA$hyd[trans_trap_no_abundance_noNA$area_type == "N"]
## W = 0.55218, p-value = 0.000131
\end{verbatim}

\begin{Shaded}
\begin{Highlighting}[]
\FunctionTok{shapiro.test}\NormalTok{(trap\_no\_abundance\_noNA}\SpecialCharTok{$}\NormalTok{hyd[trans\_trap\_no\_abundance\_noNA}\SpecialCharTok{$}\NormalTok{area\_type}\SpecialCharTok{==}\StringTok{"U"}\NormalTok{])}
\end{Highlighting}
\end{Shaded}

\begin{verbatim}
## 
##  Shapiro-Wilk normality test
## 
## data:  trap_no_abundance_noNA$hyd[trans_trap_no_abundance_noNA$area_type == "U"]
## W = 0.77091, p-value = 0.04595
\end{verbatim}

\begin{Shaded}
\begin{Highlighting}[]
\FunctionTok{shapiro.test}\NormalTok{(trap\_no\_abundance\_noNA}\SpecialCharTok{$}\NormalTok{mel[trans\_trap\_no\_abundance\_noNA}\SpecialCharTok{$}\NormalTok{area\_type}\SpecialCharTok{==}\StringTok{"N"}\NormalTok{])}
\end{Highlighting}
\end{Shaded}

\begin{verbatim}
## 
##  Shapiro-Wilk normality test
## 
## data:  trap_no_abundance_noNA$mel[trans_trap_no_abundance_noNA$area_type == "N"]
## W = 0.85588, p-value = 0.2138
\end{verbatim}

\begin{Shaded}
\begin{Highlighting}[]
\FunctionTok{shapiro.test}\NormalTok{(trap\_no\_abundance\_noNA}\SpecialCharTok{$}\NormalTok{mel[trans\_trap\_no\_abundance\_noNA}\SpecialCharTok{$}\NormalTok{area\_type}\SpecialCharTok{==}\StringTok{"U"}\NormalTok{])}
\end{Highlighting}
\end{Shaded}

\begin{verbatim}
## 
##  Shapiro-Wilk normality test
## 
## data:  trap_no_abundance_noNA$mel[trans_trap_no_abundance_noNA$area_type == "U"]
## W = 0.99896, p-value = 0.9996
\end{verbatim}

\begin{Shaded}
\begin{Highlighting}[]
\FunctionTok{shapiro.test}\NormalTok{(trap\_no\_abundance\_noNA}\SpecialCharTok{$}\NormalTok{bus[trans\_trap\_no\_abundance\_noNA}\SpecialCharTok{$}\NormalTok{area\_type}\SpecialCharTok{==}\StringTok{"N"}\NormalTok{])}
\end{Highlighting}
\end{Shaded}

\begin{verbatim}
## 
##  Shapiro-Wilk normality test
## 
## data:  trap_no_abundance_noNA$bus[trans_trap_no_abundance_noNA$area_type == "N"]
## W = 0.68403, p-value = 0.00647
\end{verbatim}

\begin{Shaded}
\begin{Highlighting}[]
\FunctionTok{shapiro.test}\NormalTok{(trap\_no\_abundance\_noNA}\SpecialCharTok{$}\NormalTok{bus[trans\_trap\_no\_abundance\_noNA}\SpecialCharTok{$}\NormalTok{area\_type}\SpecialCharTok{==}\StringTok{"U"}\NormalTok{])}
\end{Highlighting}
\end{Shaded}

\begin{verbatim}
## 
##  Shapiro-Wilk normality test
## 
## data:  trap_no_abundance_noNA$bus[trans_trap_no_abundance_noNA$area_type == "U"]
## W = 0.83274, p-value = 0.1458
\end{verbatim}

\begin{Shaded}
\begin{Highlighting}[]
\FunctionTok{shapiro.test}\NormalTok{(trap\_no\_abundance\_noNA}\SpecialCharTok{$}\NormalTok{fun[trans\_trap\_no\_abundance\_noNA}\SpecialCharTok{$}\NormalTok{area\_type}\SpecialCharTok{==}\StringTok{"N"}\NormalTok{])}
\end{Highlighting}
\end{Shaded}

\begin{verbatim}
## 
##  Shapiro-Wilk normality test
## 
## data:  trap_no_abundance_noNA$fun[trans_trap_no_abundance_noNA$area_type == "N"]
## W = 0.68403, p-value = 0.00647
\end{verbatim}

\begin{Shaded}
\begin{Highlighting}[]
\FunctionTok{shapiro.test}\NormalTok{(trap\_no\_abundance\_noNA}\SpecialCharTok{$}\NormalTok{fun[trans\_trap\_no\_abundance\_noNA}\SpecialCharTok{$}\NormalTok{area\_type}\SpecialCharTok{==}\StringTok{"U"}\NormalTok{])}
\end{Highlighting}
\end{Shaded}

\begin{verbatim}
## 
##  Shapiro-Wilk normality test
## 
## data:  trap_no_abundance_noNA$fun[trans_trap_no_abundance_noNA$area_type == "U"]
## W = 0.88104, p-value = 0.314
\end{verbatim}

\begin{Shaded}
\begin{Highlighting}[]
\FunctionTok{shapiro.test}\NormalTok{(trap\_no\_abundance\_noNA}\SpecialCharTok{$}\NormalTok{immi[trans\_trap\_no\_abundance\_noNA}\SpecialCharTok{$}\NormalTok{area\_type}\SpecialCharTok{==}\StringTok{"N"}\NormalTok{])}
\end{Highlighting}
\end{Shaded}

\begin{verbatim}
## 
##  Shapiro-Wilk normality test
## 
## data:  trap_no_abundance_noNA$immi[trans_trap_no_abundance_noNA$area_type == "N"]
## W = 0.96463, p-value = 0.8398
\end{verbatim}

\begin{Shaded}
\begin{Highlighting}[]
\FunctionTok{shapiro.test}\NormalTok{(trap\_no\_abundance\_noNA}\SpecialCharTok{$}\NormalTok{immi[trans\_trap\_no\_abundance\_noNA}\SpecialCharTok{$}\NormalTok{area\_type}\SpecialCharTok{==}\StringTok{"U"}\NormalTok{])}
\end{Highlighting}
\end{Shaded}

\begin{verbatim}
## 
##  Shapiro-Wilk normality test
## 
## data:  trap_no_abundance_noNA$immi[trans_trap_no_abundance_noNA$area_type == "U"]
## W = 0.70495, p-value = 0.01078
\end{verbatim}

\begin{Shaded}
\begin{Highlighting}[]
\FunctionTok{shapiro.test}\NormalTok{(trap\_no\_abundance\_noNA}\SpecialCharTok{$}\NormalTok{obs[trans\_trap\_no\_abundance\_noNA}\SpecialCharTok{$}\NormalTok{area\_type}\SpecialCharTok{==}\StringTok{"N"}\NormalTok{])}
\end{Highlighting}
\end{Shaded}

\begin{verbatim}
## 
##  Shapiro-Wilk normality test
## 
## data:  trap_no_abundance_noNA$obs[trans_trap_no_abundance_noNA$area_type == "N"]
## W = 0.93102, p-value = 0.6034
\end{verbatim}

\begin{Shaded}
\begin{Highlighting}[]
\FunctionTok{shapiro.test}\NormalTok{(trap\_no\_abundance\_noNA}\SpecialCharTok{$}\NormalTok{obs[trans\_trap\_no\_abundance\_noNA}\SpecialCharTok{$}\NormalTok{area\_type}\SpecialCharTok{==}\StringTok{"U"}\NormalTok{])}
\end{Highlighting}
\end{Shaded}

\begin{verbatim}
## 
##  Shapiro-Wilk normality test
## 
## data:  trap_no_abundance_noNA$obs[trans_trap_no_abundance_noNA$area_type == "U"]
## W = 0.68403, p-value = 0.00647
\end{verbatim}

\begin{Shaded}
\begin{Highlighting}[]
\FunctionTok{shapiro.test}\NormalTok{(trap\_no\_abundance\_noNA}\SpecialCharTok{$}\NormalTok{suz[trans\_trap\_no\_abundance\_noNA}\SpecialCharTok{$}\NormalTok{area\_type}\SpecialCharTok{==}\StringTok{"N"}\NormalTok{])}
\end{Highlighting}
\end{Shaded}

\begin{verbatim}
## 
##  Shapiro-Wilk normality test
## 
## data:  trap_no_abundance_noNA$suz[trans_trap_no_abundance_noNA$area_type == "N"]
## W = 0.9064, p-value = 0.4462
\end{verbatim}

\begin{Shaded}
\begin{Highlighting}[]
\FunctionTok{shapiro.test}\NormalTok{(trap\_no\_abundance\_noNA}\SpecialCharTok{$}\NormalTok{suz[trans\_trap\_no\_abundance\_noNA}\SpecialCharTok{$}\NormalTok{area\_type}\SpecialCharTok{==}\StringTok{"U"}\NormalTok{])}
\end{Highlighting}
\end{Shaded}

\begin{verbatim}
## 
##  Shapiro-Wilk normality test
## 
## data:  trap_no_abundance_noNA$suz[trans_trap_no_abundance_noNA$area_type == "U"]
## W = 0.82653, p-value = 0.131
\end{verbatim}

\begin{Shaded}
\begin{Highlighting}[]
\FunctionTok{shapiro.test}\NormalTok{(trap\_no\_abundance\_noNA}\SpecialCharTok{$}\NormalTok{sub[trans\_trap\_no\_abundance\_noNA}\SpecialCharTok{$}\NormalTok{area\_type}\SpecialCharTok{==}\StringTok{"N"}\NormalTok{])}
\end{Highlighting}
\end{Shaded}

\begin{verbatim}
## 
##  Shapiro-Wilk normality test
## 
## data:  trap_no_abundance_noNA$sub[trans_trap_no_abundance_noNA$area_type == "N"]
## W = 0.90098, p-value = 0.4153
\end{verbatim}

\begin{Shaded}
\begin{Highlighting}[]
\FunctionTok{shapiro.test}\NormalTok{(trap\_no\_abundance\_noNA}\SpecialCharTok{$}\NormalTok{sub[trans\_trap\_no\_abundance\_noNA}\SpecialCharTok{$}\NormalTok{area\_type}\SpecialCharTok{==}\StringTok{"U"}\NormalTok{])}
\end{Highlighting}
\end{Shaded}

\begin{verbatim}
## 
##  Shapiro-Wilk normality test
## 
## data:  trap_no_abundance_noNA$sub[trans_trap_no_abundance_noNA$area_type == "U"]
## W = 0.90202, p-value = 0.4211
\end{verbatim}

\begin{Shaded}
\begin{Highlighting}[]
\FunctionTok{shapiro.test}\NormalTok{(trap\_no\_abundance\_noNA}\SpecialCharTok{$}\NormalTok{tris[trans\_trap\_no\_abundance\_noNA}\SpecialCharTok{$}\NormalTok{area\_type}\SpecialCharTok{==}\StringTok{"N"}\NormalTok{])}
\end{Highlighting}
\end{Shaded}

\begin{verbatim}
## 
##  Shapiro-Wilk normality test
## 
## data:  trap_no_abundance_noNA$tris[trans_trap_no_abundance_noNA$area_type == "N"]
## W = 0.90309, p-value = 0.4272
\end{verbatim}

\begin{Shaded}
\begin{Highlighting}[]
\FunctionTok{shapiro.test}\NormalTok{(trap\_no\_abundance\_noNA}\SpecialCharTok{$}\NormalTok{tris[trans\_trap\_no\_abundance\_noNA}\SpecialCharTok{$}\NormalTok{area\_type}\SpecialCharTok{==}\StringTok{"U"}\NormalTok{])}
\end{Highlighting}
\end{Shaded}

\begin{verbatim}
## 
##  Shapiro-Wilk normality test
## 
## data:  trap_no_abundance_noNA$tris[trans_trap_no_abundance_noNA$area_type == "U"]
## W = 0.68403, p-value = 0.00647
\end{verbatim}

Then, we run Welch's two sample t-tests comparing the difference in the
abundance of each species between disturbed and undisturbed sites. We
apply transformations to achieve normality where necessary. Where
transformation to normality is not possible, we instead use a
Mann-Whitney U/Wilcoxon rank-sum test.

\begin{Shaded}
\begin{Highlighting}[]
\FunctionTok{wilcox.test}\NormalTok{(hyd }\SpecialCharTok{\textasciitilde{}}\NormalTok{ area\_type, }\AttributeTok{data =}\NormalTok{ trap\_no\_abundance\_noNA)}
\end{Highlighting}
\end{Shaded}

\begin{verbatim}
## Warning in wilcox.test.default(x = DATA[[1L]], y = DATA[[2L]], ...): cannot
## compute exact p-value with ties
\end{verbatim}

\begin{verbatim}
## 
##  Wilcoxon rank sum test with continuity correction
## 
## data:  hyd by area_type
## W = 3, p-value = 0.04198
## alternative hypothesis: true location shift is not equal to 0
\end{verbatim}

\begin{Shaded}
\begin{Highlighting}[]
\FunctionTok{t.test}\NormalTok{(mel}\SpecialCharTok{\textasciitilde{}}\NormalTok{ area\_type, }\AttributeTok{data =}\NormalTok{ trap\_no\_abundance\_noNA)}
\end{Highlighting}
\end{Shaded}

\begin{verbatim}
## 
##  Welch Two Sample t-test
## 
## data:  mel by area_type
## t = -2.7802, df = 7.917, p-value = 0.02416
## alternative hypothesis: true difference in means between group N and group U is not equal to 0
## 95 percent confidence interval:
##  -37.717447  -3.482553
## sample estimates:
## mean in group N mean in group U 
##            22.8            43.4
\end{verbatim}

\begin{Shaded}
\begin{Highlighting}[]
\FunctionTok{wilcox.test}\NormalTok{(bus}\SpecialCharTok{\textasciitilde{}}\NormalTok{ area\_type, }\AttributeTok{data =}\NormalTok{ trap\_no\_abundance\_noNA)}
\end{Highlighting}
\end{Shaded}

\begin{verbatim}
## Warning in wilcox.test.default(x = DATA[[1L]], y = DATA[[2L]], ...): cannot
## compute exact p-value with ties
\end{verbatim}

\begin{verbatim}
## 
##  Wilcoxon rank sum test with continuity correction
## 
## data:  bus by area_type
## W = 11, p-value = 0.8225
## alternative hypothesis: true location shift is not equal to 0
\end{verbatim}

\begin{Shaded}
\begin{Highlighting}[]
\FunctionTok{wilcox.test}\NormalTok{(fun}\SpecialCharTok{\textasciitilde{}}\NormalTok{ area\_type, }\AttributeTok{data =}\NormalTok{ trap\_no\_abundance\_noNA)}
\end{Highlighting}
\end{Shaded}

\begin{verbatim}
## Warning in wilcox.test.default(x = DATA[[1L]], y = DATA[[2L]], ...): cannot
## compute exact p-value with ties
\end{verbatim}

\begin{verbatim}
## 
##  Wilcoxon rank sum test with continuity correction
## 
## data:  fun by area_type
## W = 17, p-value = 0.3808
## alternative hypothesis: true location shift is not equal to 0
\end{verbatim}

\begin{Shaded}
\begin{Highlighting}[]
\FunctionTok{t.test}\NormalTok{(}\FunctionTok{log10}\NormalTok{(immi)}\SpecialCharTok{\textasciitilde{}}\NormalTok{ area\_type, }\AttributeTok{data =}\NormalTok{ trap\_no\_abundance\_noNA)}
\end{Highlighting}
\end{Shaded}

\begin{verbatim}
## 
##  Welch Two Sample t-test
## 
## data:  log10(immi) by area_type
## t = -2.3349, df = 6.6493, p-value = 0.05413
## alternative hypothesis: true difference in means between group N and group U is not equal to 0
## 95 percent confidence interval:
##  -0.737945565  0.008622942
## sample estimates:
## mean in group N mean in group U 
##        1.600846        1.965508
\end{verbatim}

\begin{Shaded}
\begin{Highlighting}[]
\FunctionTok{t.test}\NormalTok{(obs}\SpecialCharTok{\textasciitilde{}}\NormalTok{ area\_type, }\AttributeTok{data =}\NormalTok{ trap\_no\_abundance\_noNA)}
\end{Highlighting}
\end{Shaded}

\begin{verbatim}
## 
##  Welch Two Sample t-test
## 
## data:  obs by area_type
## t = 1.3914, df = 6.0486, p-value = 0.2131
## alternative hypothesis: true difference in means between group N and group U is not equal to 0
## 95 percent confidence interval:
##  -1.661387  6.061387
## sample estimates:
## mean in group N mean in group U 
##             4.4             2.2
\end{verbatim}

\begin{Shaded}
\begin{Highlighting}[]
\FunctionTok{t.test}\NormalTok{(suz}\SpecialCharTok{\textasciitilde{}}\NormalTok{ area\_type, }\AttributeTok{data =}\NormalTok{ trap\_no\_abundance\_noNA)}
\end{Highlighting}
\end{Shaded}

\begin{verbatim}
## 
##  Welch Two Sample t-test
## 
## data:  suz by area_type
## t = 0.84935, df = 5.4324, p-value = 0.4315
## alternative hypothesis: true difference in means between group N and group U is not equal to 0
## 95 percent confidence interval:
##  -39.8918  80.6918
## sample estimates:
## mean in group N mean in group U 
##            55.2            34.8
\end{verbatim}

\begin{Shaded}
\begin{Highlighting}[]
\FunctionTok{t.test}\NormalTok{(sub}\SpecialCharTok{\textasciitilde{}}\NormalTok{ area\_type, }\AttributeTok{data =}\NormalTok{ trap\_no\_abundance\_noNA)}
\end{Highlighting}
\end{Shaded}

\begin{verbatim}
## 
##  Welch Two Sample t-test
## 
## data:  sub by area_type
## t = 1.1536, df = 4.5709, p-value = 0.3054
## alternative hypothesis: true difference in means between group N and group U is not equal to 0
## 95 percent confidence interval:
##  -3.361275  8.561275
## sample estimates:
## mean in group N mean in group U 
##             5.8             3.2
\end{verbatim}

\begin{Shaded}
\begin{Highlighting}[]
\FunctionTok{t.test}\NormalTok{(}\FunctionTok{sqrt}\NormalTok{(tris)}\SpecialCharTok{\textasciitilde{}}\NormalTok{ area\_type, }\AttributeTok{data =}\NormalTok{ trap\_no\_abundance\_noNA)}
\end{Highlighting}
\end{Shaded}

\begin{verbatim}
## 
##  Welch Two Sample t-test
## 
## data:  sqrt(tris) by area_type
## t = 0.13956, df = 5.3248, p-value = 0.8941
## alternative hypothesis: true difference in means between group N and group U is not equal to 0
## 95 percent confidence interval:
##  -1.120593  1.251759
## sample estimates:
## mean in group N mean in group U 
##        1.358403        1.292820
\end{verbatim}

\hypertarget{diversity-index-t-tests}{%
\subsubsection{Diversity Index T-Tests}\label{diversity-index-t-tests}}

First, we need to calculate each of the diversity indices for each trap.
To begin, we create a purely numeric data frame, containing only the
species abundance columns.

\begin{Shaded}
\begin{Highlighting}[]
\NormalTok{trap\_numeric }\OtherTok{\textless{}{-}}\NormalTok{ trap\_no\_abundance\_noNA[, }\FunctionTok{sapply}\NormalTok{(trap\_no\_abundance\_noNA, is.numeric)]}
\end{Highlighting}
\end{Shaded}

Next, we calculate each of the diversity indices, and add them to our
`trap\_no\_abundance\_noNA' data frame as new columns.

Simpsons's Diversity Index:

\begin{Shaded}
\begin{Highlighting}[]
\NormalTok{simpsons\_values }\OtherTok{\textless{}{-}} \FunctionTok{diversity}\NormalTok{(trap\_numeric, }\AttributeTok{index =} \StringTok{"simpson"}\NormalTok{)}
\NormalTok{trap\_no\_abundance\_noNA}\SpecialCharTok{$}\NormalTok{simpsons\_index }\OtherTok{\textless{}{-}}\NormalTok{ simpsons\_values}
\end{Highlighting}
\end{Shaded}

Shannon's Diversity Index:

\begin{Shaded}
\begin{Highlighting}[]
\NormalTok{shannon\_values }\OtherTok{\textless{}{-}} \FunctionTok{diversity}\NormalTok{(trap\_numeric, }\AttributeTok{index =} \StringTok{"shannon"}\NormalTok{)}
\NormalTok{trap\_no\_abundance\_noNA}\SpecialCharTok{$}\NormalTok{shannon\_index }\OtherTok{\textless{}{-}}\NormalTok{ shannon\_values}
\end{Highlighting}
\end{Shaded}

Berger-Parker Index:

\begin{Shaded}
\begin{Highlighting}[]
\NormalTok{berger\_parker\_values }\OtherTok{\textless{}{-}} \FunctionTok{apply}\NormalTok{(trap\_numeric, }\DecValTok{1}\NormalTok{, }\ControlFlowTok{function}\NormalTok{(x) }\FunctionTok{max}\NormalTok{(x) }\SpecialCharTok{/} \FunctionTok{sum}\NormalTok{(x))}
\NormalTok{trap\_no\_abundance\_noNA}\SpecialCharTok{$}\NormalTok{berger\_parker\_index }\OtherTok{\textless{}{-}}\NormalTok{ berger\_parker\_values}
\end{Highlighting}
\end{Shaded}

McIntosh's Index:

\begin{Shaded}
\begin{Highlighting}[]
\NormalTok{mcIntosh\_values }\OtherTok{\textless{}{-}} \FunctionTok{heterogeneity}\NormalTok{(trap\_numeric, }\AttributeTok{method =} \StringTok{"mcintosh"}\NormalTok{)}
\NormalTok{trap\_no\_abundance\_noNA}\SpecialCharTok{$}\NormalTok{mcIntosh\_index }\OtherTok{\textless{}{-}}\NormalTok{ mcIntosh\_values}
\end{Highlighting}
\end{Shaded}

Next we check for normality of these indices.

\begin{Shaded}
\begin{Highlighting}[]
\FunctionTok{shapiro.test}\NormalTok{(trap\_no\_abundance\_noNA}\SpecialCharTok{$}\NormalTok{simpsons\_index[trans\_trap\_no\_abundance\_noNA}\SpecialCharTok{$}\NormalTok{area\_type}\SpecialCharTok{==}\StringTok{"U"}\NormalTok{])}
\end{Highlighting}
\end{Shaded}

\begin{verbatim}
## 
##  Shapiro-Wilk normality test
## 
## data:  trap_no_abundance_noNA$simpsons_index[trans_trap_no_abundance_noNA$area_type == "U"]
## W = 0.95496, p-value = 0.7725
\end{verbatim}

\begin{Shaded}
\begin{Highlighting}[]
\FunctionTok{shapiro.test}\NormalTok{(trap\_no\_abundance\_noNA}\SpecialCharTok{$}\NormalTok{simpsons\_index[trans\_trap\_no\_abundance\_noNA}\SpecialCharTok{$}\NormalTok{area\_type}\SpecialCharTok{==}\StringTok{"N"}\NormalTok{])}
\end{Highlighting}
\end{Shaded}

\begin{verbatim}
## 
##  Shapiro-Wilk normality test
## 
## data:  trap_no_abundance_noNA$simpsons_index[trans_trap_no_abundance_noNA$area_type == "N"]
## W = 0.86332, p-value = 0.2404
\end{verbatim}

\begin{Shaded}
\begin{Highlighting}[]
\FunctionTok{shapiro.test}\NormalTok{(trap\_no\_abundance\_noNA}\SpecialCharTok{$}\NormalTok{shannon\_index[trans\_trap\_no\_abundance\_noNA}\SpecialCharTok{$}\NormalTok{area\_type}\SpecialCharTok{==}\StringTok{"U"}\NormalTok{])}
\end{Highlighting}
\end{Shaded}

\begin{verbatim}
## 
##  Shapiro-Wilk normality test
## 
## data:  trap_no_abundance_noNA$shannon_index[trans_trap_no_abundance_noNA$area_type == "U"]
## W = 0.98091, p-value = 0.9394
\end{verbatim}

\begin{Shaded}
\begin{Highlighting}[]
\FunctionTok{shapiro.test}\NormalTok{(trap\_no\_abundance\_noNA}\SpecialCharTok{$}\NormalTok{shannon\_index[trans\_trap\_no\_abundance\_noNA}\SpecialCharTok{$}\NormalTok{area\_type}\SpecialCharTok{==}\StringTok{"N"}\NormalTok{])}
\end{Highlighting}
\end{Shaded}

\begin{verbatim}
## 
##  Shapiro-Wilk normality test
## 
## data:  trap_no_abundance_noNA$shannon_index[trans_trap_no_abundance_noNA$area_type == "N"]
## W = 0.85616, p-value = 0.2148
\end{verbatim}

\begin{Shaded}
\begin{Highlighting}[]
\FunctionTok{shapiro.test}\NormalTok{(trap\_no\_abundance\_noNA}\SpecialCharTok{$}\NormalTok{berger\_parker\_index[trans\_trap\_no\_abundance\_noNA}\SpecialCharTok{$}\NormalTok{area\_type}\SpecialCharTok{==}\StringTok{"U"}\NormalTok{])}
\end{Highlighting}
\end{Shaded}

\begin{verbatim}
## 
##  Shapiro-Wilk normality test
## 
## data:  trap_no_abundance_noNA$berger_parker_index[trans_trap_no_abundance_noNA$area_type == "U"]
## W = 0.95422, p-value = 0.7672
\end{verbatim}

\begin{Shaded}
\begin{Highlighting}[]
\FunctionTok{shapiro.test}\NormalTok{(trap\_no\_abundance\_noNA}\SpecialCharTok{$}\NormalTok{berger\_parker\_index[trans\_trap\_no\_abundance\_noNA}\SpecialCharTok{$}\NormalTok{area\_type}\SpecialCharTok{==}\StringTok{"N"}\NormalTok{])}
\end{Highlighting}
\end{Shaded}

\begin{verbatim}
## 
##  Shapiro-Wilk normality test
## 
## data:  trap_no_abundance_noNA$berger_parker_index[trans_trap_no_abundance_noNA$area_type == "N"]
## W = 0.88249, p-value = 0.3208
\end{verbatim}

\begin{Shaded}
\begin{Highlighting}[]
\FunctionTok{shapiro.test}\NormalTok{(trap\_no\_abundance\_noNA}\SpecialCharTok{$}\NormalTok{mcIntosh\_index[trans\_trap\_no\_abundance\_noNA}\SpecialCharTok{$}\NormalTok{area\_type}\SpecialCharTok{==}\StringTok{"U"}\NormalTok{])}
\end{Highlighting}
\end{Shaded}

\begin{verbatim}
## 
##  Shapiro-Wilk normality test
## 
## data:  trap_no_abundance_noNA$mcIntosh_index[trans_trap_no_abundance_noNA$area_type == "U"]
## W = 0.97255, p-value = 0.8914
\end{verbatim}

\begin{Shaded}
\begin{Highlighting}[]
\FunctionTok{shapiro.test}\NormalTok{(trap\_no\_abundance\_noNA}\SpecialCharTok{$}\NormalTok{mcIntosh\_index[trans\_trap\_no\_abundance\_noNA}\SpecialCharTok{$}\NormalTok{area\_type}\SpecialCharTok{==}\StringTok{"N"}\NormalTok{])}
\end{Highlighting}
\end{Shaded}

\begin{verbatim}
## 
##  Shapiro-Wilk normality test
## 
## data:  trap_no_abundance_noNA$mcIntosh_index[trans_trap_no_abundance_noNA$area_type == "N"]
## W = 0.8883, p-value = 0.3486
\end{verbatim}

Next we conduct the Welch's t-tests. All diversity data was normally
distributed, so no transformations were necessary.

\begin{Shaded}
\begin{Highlighting}[]
\FunctionTok{t.test}\NormalTok{((berger\_parker\_index) }\SpecialCharTok{\textasciitilde{}}\NormalTok{ area\_type, }\AttributeTok{data =}\NormalTok{ trap\_no\_abundance\_noNA)}
\end{Highlighting}
\end{Shaded}

\begin{verbatim}
## 
##  Welch Two Sample t-test
## 
## data:  (berger_parker_index) by area_type
## t = 0.87983, df = 7.9571, p-value = 0.4047
## alternative hypothesis: true difference in means between group N and group U is not equal to 0
## 95 percent confidence interval:
##  -0.1232302  0.2750443
## sample estimates:
## mean in group N mean in group U 
##       0.5928801       0.5169730
\end{verbatim}

\begin{Shaded}
\begin{Highlighting}[]
\FunctionTok{t.test}\NormalTok{((shannon\_index) }\SpecialCharTok{\textasciitilde{}}\NormalTok{ area\_type, }\AttributeTok{data =}\NormalTok{ trap\_no\_abundance\_noNA)}
\end{Highlighting}
\end{Shaded}

\begin{verbatim}
## 
##  Welch Two Sample t-test
## 
## data:  (shannon_index) by area_type
## t = -0.20073, df = 7.7846, p-value = 0.8461
## alternative hypothesis: true difference in means between group N and group U is not equal to 0
## 95 percent confidence interval:
##  -0.4160281  0.3496954
## sample estimates:
## mean in group N mean in group U 
##        1.206397        1.239564
\end{verbatim}

\begin{Shaded}
\begin{Highlighting}[]
\FunctionTok{t.test}\NormalTok{((simpsons\_index) }\SpecialCharTok{\textasciitilde{}}\NormalTok{ area\_type, }\AttributeTok{data =}\NormalTok{ trap\_no\_abundance\_noNA)}
\end{Highlighting}
\end{Shaded}

\begin{verbatim}
## 
##  Welch Two Sample t-test
## 
## data:  (simpsons_index) by area_type
## t = -0.55559, df = 7.4115, p-value = 0.5949
## alternative hypothesis: true difference in means between group N and group U is not equal to 0
## 95 percent confidence interval:
##  -0.2149122  0.1323906
## sample estimates:
## mean in group N mean in group U 
##       0.5788889       0.6201497
\end{verbatim}

\begin{Shaded}
\begin{Highlighting}[]
\FunctionTok{t.test}\NormalTok{((mcIntosh\_index) }\SpecialCharTok{\textasciitilde{}}\NormalTok{ area\_type, }\AttributeTok{data =}\NormalTok{ trap\_no\_abundance\_noNA)}
\end{Highlighting}
\end{Shaded}

\begin{verbatim}
## 
##  Welch Two Sample t-test
## 
## data:  (mcIntosh_index) by area_type
## t = -0.46444, df = 7.8613, p-value = 0.6549
## alternative hypothesis: true difference in means between group N and group U is not equal to 0
## 95 percent confidence interval:
##  -0.1717833  0.1143347
## sample estimates:
## mean in group N mean in group U 
##       0.3915911       0.4203154
\end{verbatim}

Finally, we merge this data frame containing total abundances and
diversity index values of each trap with the coordinate data. We export
this data frame to use for spatial autocorrelation analyses in ArcGIS
Pro.

\begin{Shaded}
\begin{Highlighting}[]
\NormalTok{trap\_no\_abundance\_noNA\_coord }\OtherTok{\textless{}{-}} \FunctionTok{merge}\NormalTok{(trap\_no\_abundance\_noNA, coord\_data, }\AttributeTok{by =} \StringTok{"trap\_no"}\NormalTok{, }\AttributeTok{all.x =} \ConstantTok{TRUE}\NormalTok{)}

\FunctionTok{write.csv}\NormalTok{(trap\_no\_abundance\_noNA\_coord, }\StringTok{"abundances\_coord.csv"}\NormalTok{, }\AttributeTok{row.names =} \ConstantTok{FALSE}\NormalTok{)}
\end{Highlighting}
\end{Shaded}

\hypertarget{summary-statistics}{%
\subsubsection{Summary Statistics}\label{summary-statistics}}

To report the results of the t-tests, we're interested in the
accompanying means and standard errors. The t-tests/Mann-Whitney U tests
with significant results were that of \emph{D. melanogaster} and
\emph{D. hydei}, so we find their means and standard errors to report.

\emph{D. melanogaster}:

Disturbed mean:

\begin{Shaded}
\begin{Highlighting}[]
\FunctionTok{mean}\NormalTok{(trap\_no\_abundance\_noNA}\SpecialCharTok{$}\NormalTok{mel[trap\_no\_abundance\_noNA}\SpecialCharTok{$}\NormalTok{area\_type}\SpecialCharTok{==}\StringTok{"U"}\NormalTok{])}
\end{Highlighting}
\end{Shaded}

\begin{verbatim}
## [1] 43.4
\end{verbatim}

Disturbed standard error:

\begin{Shaded}
\begin{Highlighting}[]
\FunctionTok{sd}\NormalTok{(trap\_no\_abundance\_noNA}\SpecialCharTok{$}\NormalTok{mel[trap\_no\_abundance\_noNA}\SpecialCharTok{$}\NormalTok{area\_type}\SpecialCharTok{==}\StringTok{"U"}\NormalTok{]) }\SpecialCharTok{/} \FunctionTok{sqrt}\NormalTok{(}\FunctionTok{sum}\NormalTok{(trap\_no\_abundance\_noNA}\SpecialCharTok{$}\NormalTok{area\_type}\SpecialCharTok{==}\StringTok{"U"}\NormalTok{))}
\end{Highlighting}
\end{Shaded}

\begin{verbatim}
## [1] 5.500909
\end{verbatim}

Undisturbed mean:

\begin{Shaded}
\begin{Highlighting}[]
\FunctionTok{mean}\NormalTok{(trap\_no\_abundance\_noNA}\SpecialCharTok{$}\NormalTok{mel[trap\_no\_abundance\_noNA}\SpecialCharTok{$}\NormalTok{area\_type}\SpecialCharTok{==}\StringTok{"N"}\NormalTok{])}
\end{Highlighting}
\end{Shaded}

\begin{verbatim}
## [1] 22.8
\end{verbatim}

Undisturbed standard error:

\begin{Shaded}
\begin{Highlighting}[]
\FunctionTok{sd}\NormalTok{(trap\_no\_abundance\_noNA}\SpecialCharTok{$}\NormalTok{mel[trap\_no\_abundance\_noNA}\SpecialCharTok{$}\NormalTok{area\_type}\SpecialCharTok{==}\StringTok{"N"}\NormalTok{]) }\SpecialCharTok{/} \FunctionTok{sqrt}\NormalTok{(}\FunctionTok{sum}\NormalTok{(trap\_no\_abundance\_noNA}\SpecialCharTok{$}\NormalTok{area\_type}\SpecialCharTok{==}\StringTok{"N"}\NormalTok{))}
\end{Highlighting}
\end{Shaded}

\begin{verbatim}
## [1] 4.963869
\end{verbatim}

\emph{D. hydei}:

Disturbed mean:

\begin{Shaded}
\begin{Highlighting}[]
\FunctionTok{mean}\NormalTok{(trap\_no\_abundance\_noNA}\SpecialCharTok{$}\NormalTok{hyd[trap\_no\_abundance\_noNA}\SpecialCharTok{$}\NormalTok{area\_type}\SpecialCharTok{==}\StringTok{"U"}\NormalTok{])}
\end{Highlighting}
\end{Shaded}

\begin{verbatim}
## [1] 2.8
\end{verbatim}

Disturbed standard error:

\begin{Shaded}
\begin{Highlighting}[]
\FunctionTok{sd}\NormalTok{(trap\_no\_abundance\_noNA}\SpecialCharTok{$}\NormalTok{hyd[trap\_no\_abundance\_noNA}\SpecialCharTok{$}\NormalTok{area\_type}\SpecialCharTok{==}\StringTok{"U"}\NormalTok{]) }\SpecialCharTok{/} \FunctionTok{sqrt}\NormalTok{(}\FunctionTok{sum}\NormalTok{(trap\_no\_abundance\_noNA}\SpecialCharTok{$}\NormalTok{area\_type}\SpecialCharTok{==}\StringTok{"U"}\NormalTok{))}
\end{Highlighting}
\end{Shaded}

\begin{verbatim}
## [1] 0.8
\end{verbatim}

Undisturbed mean:

\begin{Shaded}
\begin{Highlighting}[]
\FunctionTok{mean}\NormalTok{(trap\_no\_abundance\_noNA}\SpecialCharTok{$}\NormalTok{hyd[trap\_no\_abundance\_noNA}\SpecialCharTok{$}\NormalTok{area\_type}\SpecialCharTok{==}\StringTok{"N"}\NormalTok{])}
\end{Highlighting}
\end{Shaded}

\begin{verbatim}
## [1] 0.2
\end{verbatim}

Undisturbed standard error:

\begin{Shaded}
\begin{Highlighting}[]
\FunctionTok{sd}\NormalTok{(trap\_no\_abundance\_noNA}\SpecialCharTok{$}\NormalTok{hyd[trap\_no\_abundance\_noNA}\SpecialCharTok{$}\NormalTok{area\_type}\SpecialCharTok{==}\StringTok{"U"}\NormalTok{]) }\SpecialCharTok{/} \FunctionTok{sqrt}\NormalTok{(}\FunctionTok{sum}\NormalTok{(trap\_no\_abundance\_noNA}\SpecialCharTok{$}\NormalTok{area\_type}\SpecialCharTok{==}\StringTok{"U"}\NormalTok{))}
\end{Highlighting}
\end{Shaded}

\begin{verbatim}
## [1] 0.8
\end{verbatim}

\hypertarget{figures-and-tables}{%
\subsubsection{Figures and Tables}\label{figures-and-tables}}

Table 1:

First, the values where manually inputted into a Microsoft Excel
spreadsheet which we import.

\begin{Shaded}
\begin{Highlighting}[]
\NormalTok{ttest\_path }\OtherTok{\textless{}{-}} \StringTok{"./data/t\_test\_results.csv"}
\NormalTok{ttest\_data }\OtherTok{\textless{}{-}} \FunctionTok{read\_csv}\NormalTok{(ttest\_path)}
\end{Highlighting}
\end{Shaded}

\begin{verbatim}
## Rows: 6 Columns: 4
## -- Column specification --------------------------------------------------------
## Delimiter: ","
## chr (1): Species
## dbl (3): t test statistic, Degrees of freedom, p-value
## 
## i Use `spec()` to retrieve the full column specification for this data.
## i Specify the column types or set `show_col_types = FALSE` to quiet this message.
\end{verbatim}

\begin{Shaded}
\begin{Highlighting}[]
\NormalTok{wilcox\_path }\OtherTok{\textless{}{-}} \StringTok{"./data/wilcoxon\_results.csv"}
\NormalTok{wilcox\_data }\OtherTok{\textless{}{-}} \FunctionTok{read\_csv}\NormalTok{(wilcox\_path)}
\end{Highlighting}
\end{Shaded}

\begin{verbatim}
## Rows: 3 Columns: 3
## -- Column specification --------------------------------------------------------
## Delimiter: ","
## chr (1): Species
## dbl (2): W test statistic, p-value
## 
## i Use `spec()` to retrieve the full column specification for this data.
## i Specify the column types or set `show_col_types = FALSE` to quiet this message.
\end{verbatim}

Next we create two tables. One for the t-test results and another for
the Mann-Whitney U/Wilcoxon rank-sum tests.

T-tests:

\begin{Shaded}
\begin{Highlighting}[]
\NormalTok{ttest\_data }\SpecialCharTok{\%\textgreater{}\%}
\NormalTok{  flextable }\SpecialCharTok{\%\textgreater{}\%}
  \FunctionTok{width}\NormalTok{(., }\AttributeTok{width =}\NormalTok{ (}\FloatTok{6.49605}\SpecialCharTok{/}\NormalTok{(}\FunctionTok{ncol}\NormalTok{(ttest\_data)))) }\SpecialCharTok{\%\textgreater{}\%}
  \FunctionTok{italic}\NormalTok{(}\AttributeTok{italic=}\NormalTok{T, }\AttributeTok{part =} \StringTok{"body"}\NormalTok{, }\AttributeTok{j =} \StringTok{"Species"}\NormalTok{) }\SpecialCharTok{\%\textgreater{}\%}
  \FunctionTok{color}\NormalTok{(}\AttributeTok{color=}\StringTok{"\#4DAC23"}\NormalTok{, }\AttributeTok{j =} \StringTok{"p{-}value"}\NormalTok{, }\AttributeTok{i =} \DecValTok{1}\NormalTok{)}
\end{Highlighting}
\end{Shaded}

\begin{verbatim}
## Warning: fonts used in `flextable` are ignored because the `pdflatex` engine is
## used and not `xelatex` or `lualatex`. You can avoid this warning by using the
## `set_flextable_defaults(fonts_ignore=TRUE)` command or use a compatible engine
## by defining `latex_engine: xelatex` in the YAML header of the R Markdown
## document.
\end{verbatim}

\global\setlength{\Oldarrayrulewidth}{\arrayrulewidth}

\global\setlength{\Oldtabcolsep}{\tabcolsep}

\setlength{\tabcolsep}{2pt}

\renewcommand*{\arraystretch}{1.5}



\providecommand{\ascline}[3]{\noalign{\global\arrayrulewidth #1}\arrayrulecolor[HTML]{#2}\cline{#3}}

\begin{longtable}[c]{|p{1.62in}|p{1.62in}|p{1.62in}|p{1.62in}}



\ascline{1.5pt}{666666}{1-4}

\multicolumn{1}{>{\raggedright}m{\dimexpr 1.62in+0\tabcolsep}}{\textcolor[HTML]{000000}{\fontsize{11}{11}\selectfont{Species}}} & \multicolumn{1}{>{\raggedleft}m{\dimexpr 1.62in+0\tabcolsep}}{\textcolor[HTML]{000000}{\fontsize{11}{11}\selectfont{t\ test\ statistic}}} & \multicolumn{1}{>{\raggedleft}m{\dimexpr 1.62in+0\tabcolsep}}{\textcolor[HTML]{000000}{\fontsize{11}{11}\selectfont{Degrees\ of\ freedom}}} & \multicolumn{1}{>{\raggedleft}m{\dimexpr 1.62in+0\tabcolsep}}{\textcolor[HTML]{000000}{\fontsize{11}{11}\selectfont{p-value}}} \\

\ascline{1.5pt}{666666}{1-4}\endfirsthead 

\ascline{1.5pt}{666666}{1-4}

\multicolumn{1}{>{\raggedright}m{\dimexpr 1.62in+0\tabcolsep}}{\textcolor[HTML]{000000}{\fontsize{11}{11}\selectfont{Species}}} & \multicolumn{1}{>{\raggedleft}m{\dimexpr 1.62in+0\tabcolsep}}{\textcolor[HTML]{000000}{\fontsize{11}{11}\selectfont{t\ test\ statistic}}} & \multicolumn{1}{>{\raggedleft}m{\dimexpr 1.62in+0\tabcolsep}}{\textcolor[HTML]{000000}{\fontsize{11}{11}\selectfont{Degrees\ of\ freedom}}} & \multicolumn{1}{>{\raggedleft}m{\dimexpr 1.62in+0\tabcolsep}}{\textcolor[HTML]{000000}{\fontsize{11}{11}\selectfont{p-value}}} \\

\ascline{1.5pt}{666666}{1-4}\endhead



\multicolumn{1}{>{\raggedright}m{\dimexpr 1.62in+0\tabcolsep}}{\textcolor[HTML]{000000}{\fontsize{11}{11}\selectfont{\textit{D.\ melanogaster}}}} & \multicolumn{1}{>{\raggedleft}m{\dimexpr 1.62in+0\tabcolsep}}{\textcolor[HTML]{000000}{\fontsize{11}{11}\selectfont{-2.80}}} & \multicolumn{1}{>{\raggedleft}m{\dimexpr 1.62in+0\tabcolsep}}{\textcolor[HTML]{000000}{\fontsize{11}{11}\selectfont{7.92}}} & \multicolumn{1}{>{\raggedleft}m{\dimexpr 1.62in+0\tabcolsep}}{\textcolor[HTML]{4DAC23}{\fontsize{11}{11}\selectfont{0.024}}} \\





\multicolumn{1}{>{\raggedright}m{\dimexpr 1.62in+0\tabcolsep}}{\textcolor[HTML]{000000}{\fontsize{11}{11}\selectfont{\textit{D.\ immigrans}}}} & \multicolumn{1}{>{\raggedleft}m{\dimexpr 1.62in+0\tabcolsep}}{\textcolor[HTML]{000000}{\fontsize{11}{11}\selectfont{-2.34}}} & \multicolumn{1}{>{\raggedleft}m{\dimexpr 1.62in+0\tabcolsep}}{\textcolor[HTML]{000000}{\fontsize{11}{11}\selectfont{6.65}}} & \multicolumn{1}{>{\raggedleft}m{\dimexpr 1.62in+0\tabcolsep}}{\textcolor[HTML]{000000}{\fontsize{11}{11}\selectfont{0.054}}} \\





\multicolumn{1}{>{\raggedright}m{\dimexpr 1.62in+0\tabcolsep}}{\textcolor[HTML]{000000}{\fontsize{11}{11}\selectfont{\textit{D.\ obscura}}}} & \multicolumn{1}{>{\raggedleft}m{\dimexpr 1.62in+0\tabcolsep}}{\textcolor[HTML]{000000}{\fontsize{11}{11}\selectfont{1.39}}} & \multicolumn{1}{>{\raggedleft}m{\dimexpr 1.62in+0\tabcolsep}}{\textcolor[HTML]{000000}{\fontsize{11}{11}\selectfont{6.05}}} & \multicolumn{1}{>{\raggedleft}m{\dimexpr 1.62in+0\tabcolsep}}{\textcolor[HTML]{000000}{\fontsize{11}{11}\selectfont{0.213}}} \\





\multicolumn{1}{>{\raggedright}m{\dimexpr 1.62in+0\tabcolsep}}{\textcolor[HTML]{000000}{\fontsize{11}{11}\selectfont{\textit{D.\ suzukii}}}} & \multicolumn{1}{>{\raggedleft}m{\dimexpr 1.62in+0\tabcolsep}}{\textcolor[HTML]{000000}{\fontsize{11}{11}\selectfont{0.85}}} & \multicolumn{1}{>{\raggedleft}m{\dimexpr 1.62in+0\tabcolsep}}{\textcolor[HTML]{000000}{\fontsize{11}{11}\selectfont{5.43}}} & \multicolumn{1}{>{\raggedleft}m{\dimexpr 1.62in+0\tabcolsep}}{\textcolor[HTML]{000000}{\fontsize{11}{11}\selectfont{0.432}}} \\





\multicolumn{1}{>{\raggedright}m{\dimexpr 1.62in+0\tabcolsep}}{\textcolor[HTML]{000000}{\fontsize{11}{11}\selectfont{\textit{D.\ subobscura}}}} & \multicolumn{1}{>{\raggedleft}m{\dimexpr 1.62in+0\tabcolsep}}{\textcolor[HTML]{000000}{\fontsize{11}{11}\selectfont{1.15}}} & \multicolumn{1}{>{\raggedleft}m{\dimexpr 1.62in+0\tabcolsep}}{\textcolor[HTML]{000000}{\fontsize{11}{11}\selectfont{4.57}}} & \multicolumn{1}{>{\raggedleft}m{\dimexpr 1.62in+0\tabcolsep}}{\textcolor[HTML]{000000}{\fontsize{11}{11}\selectfont{0.305}}} \\





\multicolumn{1}{>{\raggedright}m{\dimexpr 1.62in+0\tabcolsep}}{\textcolor[HTML]{000000}{\fontsize{11}{11}\selectfont{\textit{D.\ tristis}}}} & \multicolumn{1}{>{\raggedleft}m{\dimexpr 1.62in+0\tabcolsep}}{\textcolor[HTML]{000000}{\fontsize{11}{11}\selectfont{0.14}}} & \multicolumn{1}{>{\raggedleft}m{\dimexpr 1.62in+0\tabcolsep}}{\textcolor[HTML]{000000}{\fontsize{11}{11}\selectfont{5.33}}} & \multicolumn{1}{>{\raggedleft}m{\dimexpr 1.62in+0\tabcolsep}}{\textcolor[HTML]{000000}{\fontsize{11}{11}\selectfont{0.894}}} \\

\ascline{1.5pt}{666666}{1-4}



\end{longtable}



\arrayrulecolor[HTML]{000000}

\global\setlength{\arrayrulewidth}{\Oldarrayrulewidth}

\global\setlength{\tabcolsep}{\Oldtabcolsep}

\renewcommand*{\arraystretch}{1}

Mann-Whitney U:

\begin{Shaded}
\begin{Highlighting}[]
\NormalTok{wilcox\_data }\SpecialCharTok{\%\textgreater{}\%}
\NormalTok{  flextable }\SpecialCharTok{\%\textgreater{}\%}
  \FunctionTok{width}\NormalTok{(., }\AttributeTok{width =}\NormalTok{ (}\FloatTok{6.49605}\SpecialCharTok{/}\NormalTok{(}\FunctionTok{ncol}\NormalTok{(wilcox\_data)))) }\SpecialCharTok{\%\textgreater{}\%}
  \FunctionTok{italic}\NormalTok{(}\AttributeTok{italic=}\NormalTok{T, }\AttributeTok{part =} \StringTok{"body"}\NormalTok{, }\AttributeTok{j =} \StringTok{"Species"}\NormalTok{) }\SpecialCharTok{\%\textgreater{}\%}
  \FunctionTok{color}\NormalTok{(}\AttributeTok{color=}\StringTok{"\#4DAC23"}\NormalTok{, }\AttributeTok{j =} \StringTok{"p{-}value"}\NormalTok{, }\AttributeTok{i =} \DecValTok{1}\NormalTok{)}
\end{Highlighting}
\end{Shaded}

\begin{verbatim}
## Warning: fonts used in `flextable` are ignored because the `pdflatex` engine is
## used and not `xelatex` or `lualatex`. You can avoid this warning by using the
## `set_flextable_defaults(fonts_ignore=TRUE)` command or use a compatible engine
## by defining `latex_engine: xelatex` in the YAML header of the R Markdown
## document.
\end{verbatim}

\global\setlength{\Oldarrayrulewidth}{\arrayrulewidth}

\global\setlength{\Oldtabcolsep}{\tabcolsep}

\setlength{\tabcolsep}{2pt}

\renewcommand*{\arraystretch}{1.5}



\providecommand{\ascline}[3]{\noalign{\global\arrayrulewidth #1}\arrayrulecolor[HTML]{#2}\cline{#3}}

\begin{longtable}[c]{|p{2.17in}|p{2.17in}|p{2.17in}}



\ascline{1.5pt}{666666}{1-3}

\multicolumn{1}{>{\raggedright}m{\dimexpr 2.17in+0\tabcolsep}}{\textcolor[HTML]{000000}{\fontsize{11}{11}\selectfont{Species}}} & \multicolumn{1}{>{\raggedleft}m{\dimexpr 2.17in+0\tabcolsep}}{\textcolor[HTML]{000000}{\fontsize{11}{11}\selectfont{W\ test\ statistic}}} & \multicolumn{1}{>{\raggedleft}m{\dimexpr 2.17in+0\tabcolsep}}{\textcolor[HTML]{000000}{\fontsize{11}{11}\selectfont{p-value}}} \\

\ascline{1.5pt}{666666}{1-3}\endfirsthead 

\ascline{1.5pt}{666666}{1-3}

\multicolumn{1}{>{\raggedright}m{\dimexpr 2.17in+0\tabcolsep}}{\textcolor[HTML]{000000}{\fontsize{11}{11}\selectfont{Species}}} & \multicolumn{1}{>{\raggedleft}m{\dimexpr 2.17in+0\tabcolsep}}{\textcolor[HTML]{000000}{\fontsize{11}{11}\selectfont{W\ test\ statistic}}} & \multicolumn{1}{>{\raggedleft}m{\dimexpr 2.17in+0\tabcolsep}}{\textcolor[HTML]{000000}{\fontsize{11}{11}\selectfont{p-value}}} \\

\ascline{1.5pt}{666666}{1-3}\endhead



\multicolumn{1}{>{\raggedright}m{\dimexpr 2.17in+0\tabcolsep}}{\textcolor[HTML]{000000}{\fontsize{11}{11}\selectfont{\textit{D.\ hydei}}}} & \multicolumn{1}{>{\raggedleft}m{\dimexpr 2.17in+0\tabcolsep}}{\textcolor[HTML]{000000}{\fontsize{11}{11}\selectfont{3}}} & \multicolumn{1}{>{\raggedleft}m{\dimexpr 2.17in+0\tabcolsep}}{\textcolor[HTML]{4DAC23}{\fontsize{11}{11}\selectfont{0.04}}} \\





\multicolumn{1}{>{\raggedright}m{\dimexpr 2.17in+0\tabcolsep}}{\textcolor[HTML]{000000}{\fontsize{11}{11}\selectfont{\textit{D.\ busckii}}}} & \multicolumn{1}{>{\raggedleft}m{\dimexpr 2.17in+0\tabcolsep}}{\textcolor[HTML]{000000}{\fontsize{11}{11}\selectfont{11}}} & \multicolumn{1}{>{\raggedleft}m{\dimexpr 2.17in+0\tabcolsep}}{\textcolor[HTML]{000000}{\fontsize{11}{11}\selectfont{0.82}}} \\





\multicolumn{1}{>{\raggedright}m{\dimexpr 2.17in+0\tabcolsep}}{\textcolor[HTML]{000000}{\fontsize{11}{11}\selectfont{\textit{D.\ funebris}}}} & \multicolumn{1}{>{\raggedleft}m{\dimexpr 2.17in+0\tabcolsep}}{\textcolor[HTML]{000000}{\fontsize{11}{11}\selectfont{17}}} & \multicolumn{1}{>{\raggedleft}m{\dimexpr 2.17in+0\tabcolsep}}{\textcolor[HTML]{000000}{\fontsize{11}{11}\selectfont{0.38}}} \\

\ascline{1.5pt}{666666}{1-3}



\end{longtable}



\arrayrulecolor[HTML]{000000}

\global\setlength{\arrayrulewidth}{\Oldarrayrulewidth}

\global\setlength{\tabcolsep}{\Oldtabcolsep}

\renewcommand*{\arraystretch}{1}

These tables were then manually combined in Microsoft word.

Figure 2:

To create a bar chart showing the mean abundances of species across
disturbed and undisturbed sites, we first need to re-organise the data
and add new columns with the standard error and mean values.

\begin{Shaded}
\begin{Highlighting}[]
\NormalTok{  trap\_no\_abundance\_noNA\_long }\OtherTok{\textless{}{-}}\NormalTok{ trap\_no\_abundance\_noNA }\SpecialCharTok{\%\textgreater{}\%}
    \FunctionTok{gather}\NormalTok{(}\AttributeTok{key =} \StringTok{"species"}\NormalTok{, }\AttributeTok{value =} \StringTok{"abundance"}\NormalTok{, }\SpecialCharTok{{-}}\NormalTok{trap\_no, }\SpecialCharTok{{-}}\NormalTok{area\_type, }\SpecialCharTok{{-}}\NormalTok{berger\_parker\_index,}\SpecialCharTok{{-}}\NormalTok{mcIntosh\_index,}\SpecialCharTok{{-}}\NormalTok{shannon\_index,}\SpecialCharTok{{-}}\NormalTok{simpsons\_index)}
  

\NormalTok{  trap\_abundance\_summary\_data }\OtherTok{\textless{}{-}}\NormalTok{ trap\_no\_abundance\_noNA\_long }\SpecialCharTok{\%\textgreater{}\%}
    \FunctionTok{group\_by}\NormalTok{(species, area\_type) }\SpecialCharTok{\%\textgreater{}\%}
    \FunctionTok{summarise}\NormalTok{(}\AttributeTok{mean\_abundance =} \FunctionTok{mean}\NormalTok{(abundance),}
              \AttributeTok{se =} \FunctionTok{sd}\NormalTok{(abundance) }\SpecialCharTok{/} \FunctionTok{sqrt}\NormalTok{(}\FunctionTok{n}\NormalTok{()))}
\end{Highlighting}
\end{Shaded}

\begin{verbatim}
## `summarise()` has grouped output by 'species'. You can override using the
## `.groups` argument.
\end{verbatim}

Then, we can plot the chart.

\begin{Shaded}
\begin{Highlighting}[]
  \FunctionTok{ggplot}\NormalTok{(trap\_abundance\_summary\_data, }\FunctionTok{aes}\NormalTok{(}\AttributeTok{x =}\NormalTok{ species, }\AttributeTok{y =}\NormalTok{ mean\_abundance, }\AttributeTok{fill =}\NormalTok{ area\_type)) }\SpecialCharTok{+}
    \FunctionTok{geom\_bar}\NormalTok{(}\AttributeTok{stat =} \StringTok{"identity"}\NormalTok{, }\AttributeTok{position =} \StringTok{"dodge"}\NormalTok{) }\SpecialCharTok{+}
    \FunctionTok{geom\_errorbar}\NormalTok{(}\FunctionTok{aes}\NormalTok{(}\AttributeTok{ymin =}\NormalTok{ mean\_abundance }\SpecialCharTok{{-}}\NormalTok{ se, }\AttributeTok{ymax =}\NormalTok{ mean\_abundance }\SpecialCharTok{+}\NormalTok{ se), }\AttributeTok{width =} \FloatTok{0.2}\NormalTok{, }\AttributeTok{position =} \FunctionTok{position\_dodge}\NormalTok{(}\AttributeTok{width =} \FloatTok{0.9}\NormalTok{)) }\SpecialCharTok{+}
    \FunctionTok{labs}\NormalTok{(}\AttributeTok{y =} \StringTok{"Mean Abundance across Trapping Sites"}\NormalTok{,}
         \AttributeTok{x =} \StringTok{"Species"}\NormalTok{,}
         \AttributeTok{fill =} \StringTok{"Area Type"}\NormalTok{) }\SpecialCharTok{+}
    \FunctionTok{scale\_fill\_manual}\NormalTok{(}\AttributeTok{values =} \FunctionTok{c}\NormalTok{(}\StringTok{"N"} \OtherTok{=} \StringTok{"\#1F77B4"}\NormalTok{, }\StringTok{"U"} \OtherTok{=} \StringTok{"\#9E2626"}\NormalTok{), }\AttributeTok{labels =} \FunctionTok{c}\NormalTok{(}\StringTok{"N"} \OtherTok{=} \StringTok{"Undisturbed"}\NormalTok{, }\StringTok{"U"} \OtherTok{=} \StringTok{"Disturbed"}\NormalTok{)) }\SpecialCharTok{+}
    \FunctionTok{scale\_x\_discrete}\NormalTok{(}\AttributeTok{labels =} \FunctionTok{c}\NormalTok{(}\StringTok{"immi"}\OtherTok{=}\StringTok{"D. immigrans"}\NormalTok{,}\StringTok{"mel"}\OtherTok{=}\StringTok{"D. melanogaster"}\NormalTok{, }\StringTok{"suz"}\OtherTok{=}\StringTok{"D. suzukii"}\NormalTok{, }\StringTok{"sub"}\OtherTok{=}\StringTok{"D. subobscura"}\NormalTok{, }\StringTok{"obs"}\OtherTok{=}\StringTok{"D. obscura"}\NormalTok{, }\StringTok{"bus"}\OtherTok{=}\StringTok{"D. busckii"}\NormalTok{, }\StringTok{"fun"}\OtherTok{=}\StringTok{"D. funebris"}\NormalTok{, }\StringTok{"hyd"}\OtherTok{=}\StringTok{"D. hydei"}\NormalTok{, }\StringTok{"tris"}\OtherTok{=}\StringTok{"D. tristis"}\NormalTok{, }\StringTok{"NA"}\OtherTok{=}\StringTok{"NA"}\NormalTok{)) }\SpecialCharTok{+}
    \FunctionTok{theme\_minimal}\NormalTok{()}
\end{Highlighting}
\end{Shaded}

\includegraphics{draft_markdown_files/figure-latex/unnamed-chunk-36-1.pdf}

Figure 3:

To create a plot containing a bar chart comparing each of the diversity
index values between disturbed and undisturbed sites, we start by
ensuring that all of the diversity indices are stored as numeric values.

\begin{Shaded}
\begin{Highlighting}[]
\NormalTok{trap\_no\_abundance\_noNA }\OtherTok{\textless{}{-}}\NormalTok{ trap\_no\_abundance\_noNA }\SpecialCharTok{\%\textgreater{}\%}
    \FunctionTok{mutate}\NormalTok{(}\FunctionTok{across}\NormalTok{(}\FunctionTok{c}\NormalTok{(simpsons\_index, shannon\_index, berger\_parker\_index, mcIntosh\_index), as.numeric))}
\end{Highlighting}
\end{Shaded}

Next, we re-organise the data and add the mean and standard errors as
new columns.

\begin{Shaded}
\begin{Highlighting}[]
\NormalTok{  summary\_diversity\_data }\OtherTok{\textless{}{-}}\NormalTok{ trap\_no\_abundance\_noNA }\SpecialCharTok{\%\textgreater{}\%}
    \FunctionTok{pivot\_longer}\NormalTok{(}\AttributeTok{cols =} \FunctionTok{c}\NormalTok{(simpsons\_index, shannon\_index, berger\_parker\_index, mcIntosh\_index),}
                 \AttributeTok{names\_to =} \StringTok{"diversity\_index"}\NormalTok{,}
                 \AttributeTok{values\_to =} \StringTok{"value"}\NormalTok{) }\SpecialCharTok{\%\textgreater{}\%}
    \FunctionTok{group\_by}\NormalTok{(area\_type, diversity\_index) }\SpecialCharTok{\%\textgreater{}\%}
    \FunctionTok{summarise}\NormalTok{(}\AttributeTok{mean\_value =} \FunctionTok{mean}\NormalTok{(value),}
              \AttributeTok{se =} \FunctionTok{sd}\NormalTok{(value) }\SpecialCharTok{/} \FunctionTok{sqrt}\NormalTok{(}\FunctionTok{n}\NormalTok{()),}
              \AttributeTok{.groups =} \StringTok{"drop"}\NormalTok{)}
\end{Highlighting}
\end{Shaded}

Now, we create individual charts for each diversity index.

Simpson's Index:

\begin{Shaded}
\begin{Highlighting}[]
\NormalTok{  plot\_simpsons }\OtherTok{\textless{}{-}} \FunctionTok{ggplot}\NormalTok{(summary\_diversity\_data }\SpecialCharTok{\%\textgreater{}\%} \FunctionTok{filter}\NormalTok{(diversity\_index }\SpecialCharTok{==} \StringTok{"simpsons\_index"}\NormalTok{), }
                          \FunctionTok{aes}\NormalTok{(}\AttributeTok{x =}\NormalTok{ area\_type, }\AttributeTok{y =}\NormalTok{ mean\_value, }\AttributeTok{fill =}\NormalTok{ area\_type)) }\SpecialCharTok{+}
    \FunctionTok{geom\_bar}\NormalTok{(}\AttributeTok{stat =} \StringTok{"identity"}\NormalTok{, }\AttributeTok{position =} \FunctionTok{position\_dodge}\NormalTok{(}\AttributeTok{width =} \FloatTok{0.9}\NormalTok{)) }\SpecialCharTok{+}
    \FunctionTok{geom\_errorbar}\NormalTok{(}\FunctionTok{aes}\NormalTok{(}\AttributeTok{ymin =}\NormalTok{ mean\_value }\SpecialCharTok{{-}}\NormalTok{ se, }\AttributeTok{ymax =}\NormalTok{ mean\_value }\SpecialCharTok{+}\NormalTok{ se), }
                  \AttributeTok{position =} \FunctionTok{position\_dodge}\NormalTok{(}\AttributeTok{width =} \FloatTok{0.9}\NormalTok{), }
                  \AttributeTok{width =} \FloatTok{0.2}\NormalTok{) }\SpecialCharTok{+}
    \FunctionTok{labs}\NormalTok{(}\AttributeTok{y =} \StringTok{"Mean Diversity Index Value"}\NormalTok{, }\AttributeTok{fill =} \StringTok{"Area Type"}\NormalTok{, }\AttributeTok{x =} \StringTok{""}\NormalTok{)}\SpecialCharTok{+}
    \FunctionTok{scale\_fill\_manual}\NormalTok{(}\AttributeTok{values =} \FunctionTok{c}\NormalTok{(}\StringTok{"N"} \OtherTok{=} \StringTok{"\#4A2354"}\NormalTok{, }\StringTok{"U"} \OtherTok{=} \StringTok{"\#BB8FCE"}\NormalTok{), }\AttributeTok{labels =} \FunctionTok{c}\NormalTok{(}\StringTok{"N"} \OtherTok{=} \StringTok{"Undisturbed"}\NormalTok{, }\StringTok{"U"} \OtherTok{=} \StringTok{"Disturbed"}\NormalTok{))}\SpecialCharTok{+}
    \FunctionTok{theme\_minimal}\NormalTok{()}\SpecialCharTok{+}
    \FunctionTok{ylim}\NormalTok{(}\DecValTok{0}\NormalTok{,}\DecValTok{1}\NormalTok{)}\SpecialCharTok{+}
    \FunctionTok{scale\_x\_discrete}\NormalTok{(}\AttributeTok{labels =} \FunctionTok{c}\NormalTok{(}\StringTok{"N"}\OtherTok{=}\StringTok{"Undisturbed"}\NormalTok{,}\StringTok{"U"}\OtherTok{=}\StringTok{"Disturbed"}\NormalTok{))}\SpecialCharTok{+}
    \FunctionTok{guides}\NormalTok{(}\AttributeTok{fill=}\StringTok{\textquotesingle{}none\textquotesingle{}}\NormalTok{)}
\end{Highlighting}
\end{Shaded}

Shannon's Index:

\begin{Shaded}
\begin{Highlighting}[]
\NormalTok{  plot\_shannon }\OtherTok{\textless{}{-}} \FunctionTok{ggplot}\NormalTok{(summary\_diversity\_data }\SpecialCharTok{\%\textgreater{}\%} \FunctionTok{filter}\NormalTok{(diversity\_index }\SpecialCharTok{==} \StringTok{"shannon\_index"}\NormalTok{), }
                         \FunctionTok{aes}\NormalTok{(}\AttributeTok{x =}\NormalTok{ area\_type, }\AttributeTok{y =}\NormalTok{ mean\_value, }\AttributeTok{fill =}\NormalTok{ area\_type)) }\SpecialCharTok{+}
    \FunctionTok{geom\_bar}\NormalTok{(}\AttributeTok{stat =} \StringTok{"identity"}\NormalTok{, }\AttributeTok{position =} \FunctionTok{position\_dodge}\NormalTok{(}\AttributeTok{width =} \FloatTok{0.9}\NormalTok{)) }\SpecialCharTok{+}
    \FunctionTok{geom\_errorbar}\NormalTok{(}\FunctionTok{aes}\NormalTok{(}\AttributeTok{ymin =}\NormalTok{ mean\_value }\SpecialCharTok{{-}}\NormalTok{ se, }\AttributeTok{ymax =}\NormalTok{ mean\_value }\SpecialCharTok{+}\NormalTok{ se), }
                  \AttributeTok{position =} \FunctionTok{position\_dodge}\NormalTok{(}\AttributeTok{width =} \FloatTok{0.9}\NormalTok{), }
                  \AttributeTok{width =} \FloatTok{0.2}\NormalTok{) }\SpecialCharTok{+}
    \FunctionTok{labs}\NormalTok{(}\AttributeTok{y =} \StringTok{"Mean Diversity Index Value"}\NormalTok{, }\AttributeTok{fill =} \StringTok{"Area Type"}\NormalTok{, }\AttributeTok{x =} \StringTok{""}\NormalTok{) }\SpecialCharTok{+}
    \FunctionTok{scale\_fill\_manual}\NormalTok{(}\AttributeTok{values =} \FunctionTok{c}\NormalTok{(}\StringTok{"N"} \OtherTok{=} \StringTok{"\#0B5345"}\NormalTok{, }\StringTok{"U"} \OtherTok{=} \StringTok{"\#73C6B6"}\NormalTok{), }\AttributeTok{labels =} \FunctionTok{c}\NormalTok{(}\StringTok{"N"} \OtherTok{=} \StringTok{"Undisturbed"}\NormalTok{, }\StringTok{"U"} \OtherTok{=} \StringTok{"Disturbed"}\NormalTok{))}\SpecialCharTok{+}
    \FunctionTok{theme\_minimal}\NormalTok{()}\SpecialCharTok{+}
    \FunctionTok{scale\_x\_discrete}\NormalTok{(}\AttributeTok{labels =} \FunctionTok{c}\NormalTok{(}\StringTok{"N"}\OtherTok{=}\StringTok{"Undisturbed"}\NormalTok{,}\StringTok{"U"}\OtherTok{=}\StringTok{"Disturbed"}\NormalTok{))}\SpecialCharTok{+}
    \FunctionTok{guides}\NormalTok{(}\AttributeTok{fill=}\StringTok{\textquotesingle{}none\textquotesingle{}}\NormalTok{)}
\end{Highlighting}
\end{Shaded}

Berger-Parker Index:

\begin{Shaded}
\begin{Highlighting}[]
\NormalTok{  plot\_berger\_parker }\OtherTok{\textless{}{-}} \FunctionTok{ggplot}\NormalTok{(summary\_diversity\_data }\SpecialCharTok{\%\textgreater{}\%} \FunctionTok{filter}\NormalTok{(diversity\_index }\SpecialCharTok{==} \StringTok{"berger\_parker\_index"}\NormalTok{), }
                               \FunctionTok{aes}\NormalTok{(}\AttributeTok{x =}\NormalTok{ area\_type, }\AttributeTok{y =}\NormalTok{ mean\_value, }\AttributeTok{fill =}\NormalTok{ area\_type)) }\SpecialCharTok{+}
    \FunctionTok{geom\_bar}\NormalTok{(}\AttributeTok{stat =} \StringTok{"identity"}\NormalTok{, }\AttributeTok{position =} \FunctionTok{position\_dodge}\NormalTok{(}\AttributeTok{width =} \FloatTok{0.9}\NormalTok{)) }\SpecialCharTok{+}
    \FunctionTok{geom\_errorbar}\NormalTok{(}\FunctionTok{aes}\NormalTok{(}\AttributeTok{ymin =}\NormalTok{ mean\_value }\SpecialCharTok{{-}}\NormalTok{ se, }\AttributeTok{ymax =}\NormalTok{ mean\_value }\SpecialCharTok{+}\NormalTok{ se), }
                  \AttributeTok{position =} \FunctionTok{position\_dodge}\NormalTok{(}\AttributeTok{width =} \FloatTok{0.9}\NormalTok{), }
                  \AttributeTok{width =} \FloatTok{0.2}\NormalTok{) }\SpecialCharTok{+}
    \FunctionTok{labs}\NormalTok{(}\AttributeTok{y =} \StringTok{"Mean Diversity Index Value"}\NormalTok{, }\AttributeTok{fill =} \StringTok{"Area Type"}\NormalTok{, }\AttributeTok{x =} \StringTok{""}\NormalTok{)}\SpecialCharTok{+}
    \FunctionTok{scale\_fill\_manual}\NormalTok{(}\AttributeTok{values =} \FunctionTok{c}\NormalTok{(}\StringTok{"N"} \OtherTok{=} \StringTok{"\#7D6608"}\NormalTok{, }\StringTok{"U"} \OtherTok{=} \StringTok{"\#F7DC6F"}\NormalTok{), }\AttributeTok{labels =} \FunctionTok{c}\NormalTok{(}\StringTok{"N"} \OtherTok{=} \StringTok{"Undisturbed"}\NormalTok{, }\StringTok{"U"} \OtherTok{=} \StringTok{"Disturbed"}\NormalTok{))}\SpecialCharTok{+}
    \FunctionTok{theme\_minimal}\NormalTok{()}\SpecialCharTok{+}
    \FunctionTok{ylim}\NormalTok{(}\DecValTok{0}\NormalTok{,}\DecValTok{1}\NormalTok{)}\SpecialCharTok{+}
    \FunctionTok{scale\_x\_discrete}\NormalTok{(}\AttributeTok{labels =} \FunctionTok{c}\NormalTok{(}\StringTok{"N"}\OtherTok{=}\StringTok{"Undisturbed"}\NormalTok{,}\StringTok{"U"}\OtherTok{=}\StringTok{"Disturbed"}\NormalTok{))}\SpecialCharTok{+}
    \FunctionTok{guides}\NormalTok{(}\AttributeTok{fill=}\StringTok{\textquotesingle{}none\textquotesingle{}}\NormalTok{)}
\end{Highlighting}
\end{Shaded}

McIntosh's Index:

\begin{Shaded}
\begin{Highlighting}[]
\NormalTok{  plot\_mcintosh }\OtherTok{\textless{}{-}} \FunctionTok{ggplot}\NormalTok{(summary\_diversity\_data }\SpecialCharTok{\%\textgreater{}\%} \FunctionTok{filter}\NormalTok{(diversity\_index }\SpecialCharTok{==} \StringTok{"mcIntosh\_index"}\NormalTok{), }
                          \FunctionTok{aes}\NormalTok{(}\AttributeTok{x =}\NormalTok{ area\_type, }\AttributeTok{y =}\NormalTok{ mean\_value, }\AttributeTok{fill =}\NormalTok{ area\_type)) }\SpecialCharTok{+}
    \FunctionTok{geom\_bar}\NormalTok{(}\AttributeTok{stat =} \StringTok{"identity"}\NormalTok{, }\AttributeTok{position =} \FunctionTok{position\_dodge}\NormalTok{(}\AttributeTok{width =} \FloatTok{0.9}\NormalTok{)) }\SpecialCharTok{+}
    \FunctionTok{geom\_errorbar}\NormalTok{(}\FunctionTok{aes}\NormalTok{(}\AttributeTok{ymin =}\NormalTok{ mean\_value }\SpecialCharTok{{-}}\NormalTok{ se, }\AttributeTok{ymax =}\NormalTok{ mean\_value }\SpecialCharTok{+}\NormalTok{ se), }
                  \AttributeTok{position =} \FunctionTok{position\_dodge}\NormalTok{(}\AttributeTok{width =} \FloatTok{0.9}\NormalTok{), }
                  \AttributeTok{width =} \FloatTok{0.2}\NormalTok{) }\SpecialCharTok{+}
    \FunctionTok{labs}\NormalTok{(}\AttributeTok{y =} \StringTok{"Mean Diversity Index Value"}\NormalTok{, }\AttributeTok{x =} \StringTok{""}\NormalTok{) }\SpecialCharTok{+}
    \FunctionTok{scale\_fill\_manual}\NormalTok{(}\AttributeTok{values =} \FunctionTok{c}\NormalTok{(}\StringTok{"N"} \OtherTok{=} \StringTok{"\#6E2C00"}\NormalTok{, }\StringTok{"U"} \OtherTok{=} \StringTok{"\#E59866"}\NormalTok{), }\AttributeTok{labels =} \FunctionTok{c}\NormalTok{(}\StringTok{"N"} \OtherTok{=} \StringTok{"Undisturbed"}\NormalTok{, }\StringTok{"U"} \OtherTok{=} \StringTok{"Disturbed"}\NormalTok{))}\SpecialCharTok{+} 
    \FunctionTok{theme\_minimal}\NormalTok{()}\SpecialCharTok{+}
    \FunctionTok{ylim}\NormalTok{(}\DecValTok{0}\NormalTok{,}\DecValTok{1}\NormalTok{)}\SpecialCharTok{+}
    \FunctionTok{scale\_x\_discrete}\NormalTok{(}\AttributeTok{labels =} \FunctionTok{c}\NormalTok{(}\StringTok{"N"}\OtherTok{=}\StringTok{"Undisturbed"}\NormalTok{,}\StringTok{"U"}\OtherTok{=}\StringTok{"Disturbed"}\NormalTok{))}\SpecialCharTok{+}
    \FunctionTok{guides}\NormalTok{(}\AttributeTok{fill=}\StringTok{\textquotesingle{}none\textquotesingle{}}\NormalTok{)}
\end{Highlighting}
\end{Shaded}

Next, we combine these charts into a single plot.

\begin{Shaded}
\begin{Highlighting}[]
\NormalTok{ combined\_plots }\OtherTok{\textless{}{-}}\NormalTok{ plot\_simpsons }\SpecialCharTok{+}\NormalTok{ plot\_shannon }\SpecialCharTok{+}\NormalTok{ plot\_berger\_parker }\SpecialCharTok{+}\NormalTok{ plot\_mcintosh}
 

\NormalTok{ combined\_plots }\SpecialCharTok{+} \FunctionTok{plot\_layout}\NormalTok{(}\AttributeTok{guides =} \StringTok{\textquotesingle{}collect\textquotesingle{}}\NormalTok{, }\AttributeTok{axes =} \StringTok{\textquotesingle{}collect\textquotesingle{}}\NormalTok{) }\SpecialCharTok{+} \FunctionTok{plot\_annotation}\NormalTok{(}\AttributeTok{tag\_levels =} \StringTok{\textquotesingle{}A\textquotesingle{}}\NormalTok{)}
\end{Highlighting}
\end{Shaded}

\includegraphics{draft_markdown_files/figure-latex/unnamed-chunk-43-1.pdf}

Table 2:

To create a table of the spatial autocorrelation (Global Moran's I)
results, data from the ArcGIS Pro analyses was entered into a Microsoft
Excel spreadsheet. We then import this.

\begin{Shaded}
\begin{Highlighting}[]
\NormalTok{autocor\_path }\OtherTok{\textless{}{-}} \StringTok{"./data/autocorrelation\_results.csv"}
\NormalTok{autocor\_data }\OtherTok{\textless{}{-}} \FunctionTok{read\_csv}\NormalTok{(autocor\_path)}
\end{Highlighting}
\end{Shaded}

\begin{verbatim}
## Rows: 9 Columns: 6
## -- Column specification --------------------------------------------------------
## Delimiter: ","
## chr (1): Species
## dbl (5): Moran's Index, Expected Index, Variance, z-score, p-value
## 
## i Use `spec()` to retrieve the full column specification for this data.
## i Specify the column types or set `show_col_types = FALSE` to quiet this message.
\end{verbatim}

Next, we create a table using this data.

\begin{Shaded}
\begin{Highlighting}[]
\NormalTok{autocor\_data }\SpecialCharTok{\%\textgreater{}\%}
\NormalTok{  flextable }\SpecialCharTok{\%\textgreater{}\%}
  \FunctionTok{width}\NormalTok{(., }\AttributeTok{width =}\NormalTok{ (}\FloatTok{6.49605}\SpecialCharTok{/}\NormalTok{(}\FunctionTok{ncol}\NormalTok{(autocor\_data)))) }\SpecialCharTok{\%\textgreater{}\%}
  \FunctionTok{italic}\NormalTok{(}\AttributeTok{italic=}\NormalTok{T, }\AttributeTok{part =} \StringTok{"body"}\NormalTok{, }\AttributeTok{j =} \StringTok{"Species"}\NormalTok{) }\SpecialCharTok{\%\textgreater{}\%}
  \FunctionTok{color}\NormalTok{(}\AttributeTok{color=}\StringTok{"\#4DAC23"}\NormalTok{, }\AttributeTok{j =} \StringTok{"p{-}value"}\NormalTok{, }\AttributeTok{i =} \DecValTok{4}\NormalTok{)}
\end{Highlighting}
\end{Shaded}

\begin{verbatim}
## Warning: fonts used in `flextable` are ignored because the `pdflatex` engine is
## used and not `xelatex` or `lualatex`. You can avoid this warning by using the
## `set_flextable_defaults(fonts_ignore=TRUE)` command or use a compatible engine
## by defining `latex_engine: xelatex` in the YAML header of the R Markdown
## document.
\end{verbatim}

\global\setlength{\Oldarrayrulewidth}{\arrayrulewidth}

\global\setlength{\Oldtabcolsep}{\tabcolsep}

\setlength{\tabcolsep}{2pt}

\renewcommand*{\arraystretch}{1.5}



\providecommand{\ascline}[3]{\noalign{\global\arrayrulewidth #1}\arrayrulecolor[HTML]{#2}\cline{#3}}

\begin{longtable}[c]{|p{1.08in}|p{1.08in}|p{1.08in}|p{1.08in}|p{1.08in}|p{1.08in}}



\ascline{1.5pt}{666666}{1-6}

\multicolumn{1}{>{\raggedright}m{\dimexpr 1.08in+0\tabcolsep}}{\textcolor[HTML]{000000}{\fontsize{11}{11}\selectfont{Species}}} & \multicolumn{1}{>{\raggedleft}m{\dimexpr 1.08in+0\tabcolsep}}{\textcolor[HTML]{000000}{\fontsize{11}{11}\selectfont{Moran's\ Index}}} & \multicolumn{1}{>{\raggedleft}m{\dimexpr 1.08in+0\tabcolsep}}{\textcolor[HTML]{000000}{\fontsize{11}{11}\selectfont{Expected\ Index}}} & \multicolumn{1}{>{\raggedleft}m{\dimexpr 1.08in+0\tabcolsep}}{\textcolor[HTML]{000000}{\fontsize{11}{11}\selectfont{Variance}}} & \multicolumn{1}{>{\raggedleft}m{\dimexpr 1.08in+0\tabcolsep}}{\textcolor[HTML]{000000}{\fontsize{11}{11}\selectfont{z-score}}} & \multicolumn{1}{>{\raggedleft}m{\dimexpr 1.08in+0\tabcolsep}}{\textcolor[HTML]{000000}{\fontsize{11}{11}\selectfont{p-value}}} \\

\ascline{1.5pt}{666666}{1-6}\endfirsthead 

\ascline{1.5pt}{666666}{1-6}

\multicolumn{1}{>{\raggedright}m{\dimexpr 1.08in+0\tabcolsep}}{\textcolor[HTML]{000000}{\fontsize{11}{11}\selectfont{Species}}} & \multicolumn{1}{>{\raggedleft}m{\dimexpr 1.08in+0\tabcolsep}}{\textcolor[HTML]{000000}{\fontsize{11}{11}\selectfont{Moran's\ Index}}} & \multicolumn{1}{>{\raggedleft}m{\dimexpr 1.08in+0\tabcolsep}}{\textcolor[HTML]{000000}{\fontsize{11}{11}\selectfont{Expected\ Index}}} & \multicolumn{1}{>{\raggedleft}m{\dimexpr 1.08in+0\tabcolsep}}{\textcolor[HTML]{000000}{\fontsize{11}{11}\selectfont{Variance}}} & \multicolumn{1}{>{\raggedleft}m{\dimexpr 1.08in+0\tabcolsep}}{\textcolor[HTML]{000000}{\fontsize{11}{11}\selectfont{z-score}}} & \multicolumn{1}{>{\raggedleft}m{\dimexpr 1.08in+0\tabcolsep}}{\textcolor[HTML]{000000}{\fontsize{11}{11}\selectfont{p-value}}} \\

\ascline{1.5pt}{666666}{1-6}\endhead



\multicolumn{1}{>{\raggedright}m{\dimexpr 1.08in+0\tabcolsep}}{\textcolor[HTML]{000000}{\fontsize{11}{11}\selectfont{\textit{D.\ busckii}}}} & \multicolumn{1}{>{\raggedleft}m{\dimexpr 1.08in+0\tabcolsep}}{\textcolor[HTML]{000000}{\fontsize{11}{11}\selectfont{-0.35}}} & \multicolumn{1}{>{\raggedleft}m{\dimexpr 1.08in+0\tabcolsep}}{\textcolor[HTML]{000000}{\fontsize{11}{11}\selectfont{-0.11}}} & \multicolumn{1}{>{\raggedleft}m{\dimexpr 1.08in+0\tabcolsep}}{\textcolor[HTML]{000000}{\fontsize{11}{11}\selectfont{0.04}}} & \multicolumn{1}{>{\raggedleft}m{\dimexpr 1.08in+0\tabcolsep}}{\textcolor[HTML]{000000}{\fontsize{11}{11}\selectfont{-1.26}}} & \multicolumn{1}{>{\raggedleft}m{\dimexpr 1.08in+0\tabcolsep}}{\textcolor[HTML]{000000}{\fontsize{11}{11}\selectfont{0.206}}} \\





\multicolumn{1}{>{\raggedright}m{\dimexpr 1.08in+0\tabcolsep}}{\textcolor[HTML]{000000}{\fontsize{11}{11}\selectfont{\textit{D.\ funebris}}}} & \multicolumn{1}{>{\raggedleft}m{\dimexpr 1.08in+0\tabcolsep}}{\textcolor[HTML]{000000}{\fontsize{11}{11}\selectfont{0.19}}} & \multicolumn{1}{>{\raggedleft}m{\dimexpr 1.08in+0\tabcolsep}}{\textcolor[HTML]{000000}{\fontsize{11}{11}\selectfont{-0.11}}} & \multicolumn{1}{>{\raggedleft}m{\dimexpr 1.08in+0\tabcolsep}}{\textcolor[HTML]{000000}{\fontsize{11}{11}\selectfont{0.04}}} & \multicolumn{1}{>{\raggedleft}m{\dimexpr 1.08in+0\tabcolsep}}{\textcolor[HTML]{000000}{\fontsize{11}{11}\selectfont{1.55}}} & \multicolumn{1}{>{\raggedleft}m{\dimexpr 1.08in+0\tabcolsep}}{\textcolor[HTML]{000000}{\fontsize{11}{11}\selectfont{0.121}}} \\





\multicolumn{1}{>{\raggedright}m{\dimexpr 1.08in+0\tabcolsep}}{\textcolor[HTML]{000000}{\fontsize{11}{11}\selectfont{\textit{D.\ hydei}}}} & \multicolumn{1}{>{\raggedleft}m{\dimexpr 1.08in+0\tabcolsep}}{\textcolor[HTML]{000000}{\fontsize{11}{11}\selectfont{-0.05}}} & \multicolumn{1}{>{\raggedleft}m{\dimexpr 1.08in+0\tabcolsep}}{\textcolor[HTML]{000000}{\fontsize{11}{11}\selectfont{-0.11}}} & \multicolumn{1}{>{\raggedleft}m{\dimexpr 1.08in+0\tabcolsep}}{\textcolor[HTML]{000000}{\fontsize{11}{11}\selectfont{0.04}}} & \multicolumn{1}{>{\raggedleft}m{\dimexpr 1.08in+0\tabcolsep}}{\textcolor[HTML]{000000}{\fontsize{11}{11}\selectfont{0.33}}} & \multicolumn{1}{>{\raggedleft}m{\dimexpr 1.08in+0\tabcolsep}}{\textcolor[HTML]{000000}{\fontsize{11}{11}\selectfont{0.741}}} \\





\multicolumn{1}{>{\raggedright}m{\dimexpr 1.08in+0\tabcolsep}}{\textcolor[HTML]{000000}{\fontsize{11}{11}\selectfont{\textit{D.\ immigrans}}}} & \multicolumn{1}{>{\raggedleft}m{\dimexpr 1.08in+0\tabcolsep}}{\textcolor[HTML]{000000}{\fontsize{11}{11}\selectfont{0.10}}} & \multicolumn{1}{>{\raggedleft}m{\dimexpr 1.08in+0\tabcolsep}}{\textcolor[HTML]{000000}{\fontsize{11}{11}\selectfont{-0.11}}} & \multicolumn{1}{>{\raggedleft}m{\dimexpr 1.08in+0\tabcolsep}}{\textcolor[HTML]{000000}{\fontsize{11}{11}\selectfont{0.01}}} & \multicolumn{1}{>{\raggedleft}m{\dimexpr 1.08in+0\tabcolsep}}{\textcolor[HTML]{000000}{\fontsize{11}{11}\selectfont{2.10}}} & \multicolumn{1}{>{\raggedleft}m{\dimexpr 1.08in+0\tabcolsep}}{\textcolor[HTML]{4DAC23}{\fontsize{11}{11}\selectfont{0.036}}} \\





\multicolumn{1}{>{\raggedright}m{\dimexpr 1.08in+0\tabcolsep}}{\textcolor[HTML]{000000}{\fontsize{11}{11}\selectfont{\textit{D.\ melanogaster}}}} & \multicolumn{1}{>{\raggedleft}m{\dimexpr 1.08in+0\tabcolsep}}{\textcolor[HTML]{000000}{\fontsize{11}{11}\selectfont{0.24}}} & \multicolumn{1}{>{\raggedleft}m{\dimexpr 1.08in+0\tabcolsep}}{\textcolor[HTML]{000000}{\fontsize{11}{11}\selectfont{-0.11}}} & \multicolumn{1}{>{\raggedleft}m{\dimexpr 1.08in+0\tabcolsep}}{\textcolor[HTML]{000000}{\fontsize{11}{11}\selectfont{0.03}}} & \multicolumn{1}{>{\raggedleft}m{\dimexpr 1.08in+0\tabcolsep}}{\textcolor[HTML]{000000}{\fontsize{11}{11}\selectfont{1.93}}} & \multicolumn{1}{>{\raggedleft}m{\dimexpr 1.08in+0\tabcolsep}}{\textcolor[HTML]{000000}{\fontsize{11}{11}\selectfont{0.053}}} \\





\multicolumn{1}{>{\raggedright}m{\dimexpr 1.08in+0\tabcolsep}}{\textcolor[HTML]{000000}{\fontsize{11}{11}\selectfont{\textit{D.\ obscura}}}} & \multicolumn{1}{>{\raggedleft}m{\dimexpr 1.08in+0\tabcolsep}}{\textcolor[HTML]{000000}{\fontsize{11}{11}\selectfont{-0.07}}} & \multicolumn{1}{>{\raggedleft}m{\dimexpr 1.08in+0\tabcolsep}}{\textcolor[HTML]{000000}{\fontsize{11}{11}\selectfont{-0.11}}} & \multicolumn{1}{>{\raggedleft}m{\dimexpr 1.08in+0\tabcolsep}}{\textcolor[HTML]{000000}{\fontsize{11}{11}\selectfont{0.03}}} & \multicolumn{1}{>{\raggedleft}m{\dimexpr 1.08in+0\tabcolsep}}{\textcolor[HTML]{000000}{\fontsize{11}{11}\selectfont{0.23}}} & \multicolumn{1}{>{\raggedleft}m{\dimexpr 1.08in+0\tabcolsep}}{\textcolor[HTML]{000000}{\fontsize{11}{11}\selectfont{0.818}}} \\





\multicolumn{1}{>{\raggedright}m{\dimexpr 1.08in+0\tabcolsep}}{\textcolor[HTML]{000000}{\fontsize{11}{11}\selectfont{\textit{D.\ subobscura}}}} & \multicolumn{1}{>{\raggedleft}m{\dimexpr 1.08in+0\tabcolsep}}{\textcolor[HTML]{000000}{\fontsize{11}{11}\selectfont{-0.10}}} & \multicolumn{1}{>{\raggedleft}m{\dimexpr 1.08in+0\tabcolsep}}{\textcolor[HTML]{000000}{\fontsize{11}{11}\selectfont{-0.11}}} & \multicolumn{1}{>{\raggedleft}m{\dimexpr 1.08in+0\tabcolsep}}{\textcolor[HTML]{000000}{\fontsize{11}{11}\selectfont{0.03}}} & \multicolumn{1}{>{\raggedleft}m{\dimexpr 1.08in+0\tabcolsep}}{\textcolor[HTML]{000000}{\fontsize{11}{11}\selectfont{0.07}}} & \multicolumn{1}{>{\raggedleft}m{\dimexpr 1.08in+0\tabcolsep}}{\textcolor[HTML]{000000}{\fontsize{11}{11}\selectfont{0.944}}} \\





\multicolumn{1}{>{\raggedright}m{\dimexpr 1.08in+0\tabcolsep}}{\textcolor[HTML]{000000}{\fontsize{11}{11}\selectfont{\textit{D.\ suzukii}}}} & \multicolumn{1}{>{\raggedleft}m{\dimexpr 1.08in+0\tabcolsep}}{\textcolor[HTML]{000000}{\fontsize{11}{11}\selectfont{-0.12}}} & \multicolumn{1}{>{\raggedleft}m{\dimexpr 1.08in+0\tabcolsep}}{\textcolor[HTML]{000000}{\fontsize{11}{11}\selectfont{-0.11}}} & \multicolumn{1}{>{\raggedleft}m{\dimexpr 1.08in+0\tabcolsep}}{\textcolor[HTML]{000000}{\fontsize{11}{11}\selectfont{0.03}}} & \multicolumn{1}{>{\raggedleft}m{\dimexpr 1.08in+0\tabcolsep}}{\textcolor[HTML]{000000}{\fontsize{11}{11}\selectfont{-0.05}}} & \multicolumn{1}{>{\raggedleft}m{\dimexpr 1.08in+0\tabcolsep}}{\textcolor[HTML]{000000}{\fontsize{11}{11}\selectfont{0.960}}} \\





\multicolumn{1}{>{\raggedright}m{\dimexpr 1.08in+0\tabcolsep}}{\textcolor[HTML]{000000}{\fontsize{11}{11}\selectfont{\textit{D.\ tristis}}}} & \multicolumn{1}{>{\raggedleft}m{\dimexpr 1.08in+0\tabcolsep}}{\textcolor[HTML]{000000}{\fontsize{11}{11}\selectfont{-0.24}}} & \multicolumn{1}{>{\raggedleft}m{\dimexpr 1.08in+0\tabcolsep}}{\textcolor[HTML]{000000}{\fontsize{11}{11}\selectfont{-0.11}}} & \multicolumn{1}{>{\raggedleft}m{\dimexpr 1.08in+0\tabcolsep}}{\textcolor[HTML]{000000}{\fontsize{11}{11}\selectfont{0.02}}} & \multicolumn{1}{>{\raggedleft}m{\dimexpr 1.08in+0\tabcolsep}}{\textcolor[HTML]{000000}{\fontsize{11}{11}\selectfont{-0.91}}} & \multicolumn{1}{>{\raggedleft}m{\dimexpr 1.08in+0\tabcolsep}}{\textcolor[HTML]{000000}{\fontsize{11}{11}\selectfont{0.364}}} \\

\ascline{1.5pt}{666666}{1-6}



\end{longtable}



\arrayrulecolor[HTML]{000000}

\global\setlength{\arrayrulewidth}{\Oldarrayrulewidth}

\global\setlength{\tabcolsep}{\Oldtabcolsep}

\renewcommand*{\arraystretch}{1}

\end{document}
